\documentclass[10pt,a4paper]{article}
\usepackage[utf8]{inputenc}
\usepackage[ngerman]{babel}
\usepackage{amsmath}
\usepackage{amsfonts}
\usepackage{amssymb}
\usepackage{graphicx}
\usepackage{hyperref}

\usepackage{float}

\usepackage{amsthm}
\usepackage{algorithm2e}

\usepackage{tikz}
\usepackage{cancel}
\usepackage{soul}

\usepackage{setspace}
\begin{document}
\begin{center}
	{\huge \textbf{Aufgaben und Lösungen}}
\end{center}



%%%%%%%%%%%%%%%%%%%%%
%		Frage		%
%%%%%%%%%%%%%%%%%%%%%
\section*{Aufgabe 1}
8x vorgekommen \\
Beschreiben Sie jeweils eine Lösung für das Union-Find-Problem mit Laufzeit
\begin{enumerate}
	\item $O$(log $n$) (amortisiert) für UNION und O(1) für FIND
	\item O(1) für UNION und O(log$n$) für FIND
\end{enumerate}	
wobei $n$ die Anzahl der Elemente ist. Begründen Sie in beiden Fällen die entsprechenden Laufzeiten.

\subsection*{Lösung}
\subsubsection*{O(log $n$) (amortisiert) für UNION und O(1) für FIND}

\begin{tabular}{ll}
	Feld name[1...n]:& name[x] = Name des Blocks der x enthält. 1 $\leq$ x $\leq$ n \\
	size[1..n]:& size[A] = Anzahl Elemente im Block A, initialisiert mit 1 \\ 
	L[1..n]: & L[A] = Liste aller Elemente in Block A, initialisiert L[i] = \{i\}
\end{tabular}
\begin{algorithm}[H]
	
	\textbf{Initialisierung:}	\\
	\Begin{
		\For{i := 1 to n}{
			name[i]=i \\
			size[i]=1 \\ 
			L[i]=\{i\}
		}
	}
	~\\
	\textbf{FIND(x):} \\
	\Begin{
		\textbf{return} name[x] \\
	}
	~\\
	\textbf{UNION(A,B):} \\
	\Begin{
		\uIf{size[A] $\leq$ size[B]}
		{
			\ForEach{i in L[A]}{
				name[i] = B \\ 
			}
			size[B] += size[A] \\ 
			L[B] = L[B].concat(L[A]) \\ 
		}
		\Else{
			\ForEach{i in L[B]}{
				name[i] = A \\ 
			}
			size[A] += size[B] \\ 
			L[A] = L[A].concat(L[B]) \\ 
		}
	}
	
	
\end{algorithm}
\paragraph*{Laufzeit:}
FIND(x): O(1) $\rightarrow$ Einfacher Zugriff auf ein Feld\\ 
UNION: O(log $n$) $\rightarrow$ x kann maximal log(n) mal seinen Namen ändern, da es sich nach jeder Namensänderung in einer doppelt so großen Menge befindet. (Die kleinere Menge wird umbenannt)


\subsubsection*{O(1) für UNION und O(log $n$) für FIND}
\begin{tabular}{ll}
	Feld name[1...n]:& name[x] = Name des Blocks mit Wurzel x (hat nur Bedeutung, falls x Wurzel) \\
	Feld vater[1...n]:& vater[x] = $\begin{cases}
	Vater\ von\ x\ in\ seinem\ Baum\\
	0,\ falls\ x\ Wurzel
	\end{cases}$ \\
	Feld wurzel[1...n]:& wurzel[x] = Wurzel des Blocks mit Namen x \\
	Feld size[1..n]:& size[x] = Anzahl Knoten im Unterbaum mit Wurzel x \\
\end{tabular}
\begin{algorithm}[H]
	\textbf{Initialisierung:}	\\
	\Begin{
		\For{i := 1 to n}{
			vater[i]=0 \\
			name[i]=i \\
			wurzel[i]=i \\
		}
	}
	~\\
	\textbf{FIND(x):} \\
	\Begin{
		\While{vater[x] != 0}{
			x = vater[x]
		}
		\textbf{return} name[x] \\
	}
	~\\
	
	\textbf{UNION(A,B,C):} \\
	\Begin{
		$r_{1}$ = wurzel[A]	\\
		$r_{2}$ = wurzel[B]	\\
		\uIf{size[$r_{1}$] $\leq$ size[$r_{2}$]}{
			vater[$r_{1}$] = $r_{2}$	\\
			name[$r_{2}$] = C	\\
			wurzel[C] = $r_{2}$	\\ 
			size[$r_{2}$] += size[$r_{1}$] \\ 
		}
		\Else{
			vater[$r_{2}$] = $r_{1}$	\\
			name[$r_{1}$] = C	\\
			wurzel[C] = $r_{1}$	\\ 
			size[$r_{1}$] += size[$r_{2}$] \\ 
		}
	}
	
\end{algorithm}

\paragraph*{Laufzeit:}~\\
FIND(x): O(log $n$) $\rightarrow$ Weighted UNION Rule \\ 
UNION: O(1) $\rightarrow$ Nur Pointer ändern

\paragraph{Warum hat der Baum logarithmische Höhe/Tiefe?}
Im Worst-Case wird ein UNION auf zwei gleich große und gleich tiefe Bäume ausgeführt. Dabei ist die Größe von C doppelt so groß wie die ursprünglichen Bäume, jedoch ist die Tiefe nur um 1 gewachsen ($\log(size(x))\geq Hoehe(x)$)





%%%%%%%%%%%%%%%%%%%%%
%		Frage		%
%%%%%%%%%%%%%%%%%%%%%

\section*{Aufgabe 2}\label{q:1} 
8x vorgekommen \\
Entwickeln Sie eine Datenstruktur zur Speicherung von $n$ Schlüsseln aus dem Universum $\{1,...,N\}$(wobei $n<<N$), die eine Zugriffszeit von O(1) garantiert. Sie dürfen dabei $O(n^2)$ Speicherplatz verwenden.

~\\
5x vorgekommen \\
(Perfektes Hashing) Verbessern Sie die Datenstruktur aus Aufgabe \ref{q:1}, so dass nur noch Speicherplatz $O(n)$ benutzt wird.

Hashig durch Verkettung und mit offener Adressierung (Linear Probing:Wie funktioniert Delete())

\subsection*{Lösung}
\subsubsection*{Hashing mit Verkettung} löse Kollisionen nicht auf, speichere mehrere Schlüssel an der gleichen Position
\\

Speichere für jedes Ergebnis der Hashfunktion $h$ eine Liste

\textbf{Lookup(x)}: lineare Suche in Liste $T[h(x)]$ \\ 
\begin{itemize}
	\item Worst Case: alle Keys in derselben List $\rightarrow$ O(n)
	\item erwartete Zeit: O($\frac{n}{m}$)
	\item Belegungsfactor $\beta = \frac{n}{m}$ $\leftarrow$ erw. Länge einer Liste T[x]
	\item wenn $m \geq n$, d.h. $\beta \leq 1$ dann $\rightarrow$ erw. Laufzeit O(1)
\end{itemize}


\textbf{Insert(x)}: $x\notin S.$ Füge x an erst freie Stelle in $T[h(x)]$ ein

\textbf{Delete(x)}: Entferne x aus $T[h(x)]$

\begin{center}
	\resizebox{.6\columnwidth}{!}{
		

\tikzset{every picture/.style={line width=0.75pt}} %set default line width to 0.75pt        

\begin{tikzpicture}[x=0.75pt,y=0.75pt,yscale=-1,xscale=1][H]
%uncomment if require: \path (0,181); %set diagram left start at 0, and has height of 181

%Shape: Rectangle [id:dp31075993805350643] 
\draw   (20,20) -- (50,20) -- (50,40) -- (20,40) -- cycle ;
%Shape: Rectangle [id:dp07786860939393403] 
\draw   (20,60) -- (50,60) -- (50,80) -- (20,80) -- cycle ;
%Shape: Rectangle [id:dp4591623739856949] 
\draw   (20,40) -- (50,40) -- (50,60) -- (20,60) -- cycle ;
%Shape: Rectangle [id:dp983416511512617] 
\draw   (20,80) -- (50,80) -- (50,100) -- (20,100) -- cycle ;
%Shape: Rectangle [id:dp04528718387197084] 
\draw   (80,40) -- (110,40) -- (110,60) -- (80,60) -- cycle ;
%Shape: Rectangle [id:dp1270001423171485] 
\draw   (20,140) -- (50,140) -- (50,160) -- (20,160) -- cycle ;
%Shape: Rectangle [id:dp776903324067578] 
\draw   (20,120) -- (50,120) -- (50,140) -- (20,140) -- cycle ;
%Shape: Rectangle [id:dp08008497435086515] 
\draw   (20,100) -- (50,100) -- (50,120) -- (20,120) -- cycle ;
%Shape: Rectangle [id:dp6528086235651709] 
\draw   (110,40) -- (140,40) -- (140,60) -- (110,60) -- cycle ;
%Shape: Rectangle [id:dp6043880104015285] 
\draw   (170,40.5) -- (200,40.5) -- (200,60.5) -- (170,60.5) -- cycle ;
%Shape: Rectangle [id:dp5461726616378066] 
\draw   (200,40.5) -- (230,40.5) -- (230,60.5) -- (200,60.5) -- cycle ;
%Shape: Rectangle [id:dp31723847172342756] 
\draw   (80,100) -- (110,100) -- (110,120) -- (80,120) -- cycle ;
%Shape: Rectangle [id:dp9818861995587984] 
\draw   (110,100) -- (140,100) -- (140,120) -- (110,120) -- cycle ;
%Straight Lines [id:da5730569526468681] 
\draw    (35,50) -- (77,50) ;
\draw [shift={(80,50)}, rotate = 180] [fill={rgb, 255:red, 0; green, 0; blue, 0 }  ][line width=0.08]  [draw opacity=0] (8.93,-4.29) -- (0,0) -- (8.93,4.29) -- cycle    ;
\draw [shift={(35,50)}, rotate = 0] [color={rgb, 255:red, 0; green, 0; blue, 0 }  ][fill={rgb, 255:red, 0; green, 0; blue, 0 }  ][line width=0.75]      (0, 0) circle [x radius= 3.35, y radius= 3.35]   ;
%Straight Lines [id:da6922082890770056] 
\draw    (125,50) -- (167,50) ;
\draw [shift={(170,50)}, rotate = 180] [fill={rgb, 255:red, 0; green, 0; blue, 0 }  ][line width=0.08]  [draw opacity=0] (8.93,-4.29) -- (0,0) -- (8.93,4.29) -- cycle    ;
\draw [shift={(125,50)}, rotate = 0] [color={rgb, 255:red, 0; green, 0; blue, 0 }  ][fill={rgb, 255:red, 0; green, 0; blue, 0 }  ][line width=0.75]      (0, 0) circle [x radius= 3.35, y radius= 3.35]   ;
%Straight Lines [id:da7299557815790314] 
\draw    (35,110) -- (77,110) ;
\draw [shift={(80,110)}, rotate = 180] [fill={rgb, 255:red, 0; green, 0; blue, 0 }  ][line width=0.08]  [draw opacity=0] (8.93,-4.29) -- (0,0) -- (8.93,4.29) -- cycle    ;
\draw [shift={(35,110)}, rotate = 0] [color={rgb, 255:red, 0; green, 0; blue, 0 }  ][fill={rgb, 255:red, 0; green, 0; blue, 0 }  ][line width=0.75]      (0, 0) circle [x radius= 3.35, y radius= 3.35]   ;
%Straight Lines [id:da4088411111836039] 
\draw    (125,110) -- (160,110) -- (160,130) ;
\draw [shift={(160,130)}, rotate = 270] [color={rgb, 255:red, 0; green, 0; blue, 0 }  ][line width=0.75]    (0,5.59) -- (0,-5.59)   ;

%Straight Lines [id:da04469814123042082] 
\draw    (215,50) -- (250,50) -- (250,70) ;
\draw [shift={(250,70)}, rotate = 270] [color={rgb, 255:red, 0; green, 0; blue, 0 }  ][line width=0.75]    (0,5.59) -- (0,-5.59)   ;


% Text Node
\draw (13.5,50.5) node   [align=left] {2};
% Text Node
\draw (13.5,30.5) node   [align=left] {1};
% Text Node
\draw (13.5,70.5) node   [align=left] {3};
% Text Node
\draw (13.5,90.5) node   [align=left] {4};
% Text Node
\draw (13.5,110.5) node   [align=left] {5};
% Text Node
\draw (13.5,130.5) node   [align=left] {6};
% Text Node
\draw (13.5,149.5) node   [align=left] {7};
% Text Node
\draw (95,50) node   [align=left] {3};
% Text Node
\draw (185,50.5) node   [align=left] {17};
% Text Node
\draw (95,110) node   [align=left] {19};


\end{tikzpicture}

	}
\end{center}
meist wird als Hashfunktion einfaches Modulo verwendet.

\paragraph*{Verbesserung Verdopplungs-Strategie:}
\begin{itemize}
	\item Immer wenn $\beta>2$, verdopple Tafelgröße $\rightarrow1$ sehr teures Insert (da alle Elemente mit neuer Hashfunktion umgespeichert werden), im Schnitt aber weiter O(1)\\ 
	\item Bei Delete und kleinem $\beta$: Tabelle kann halbiert werden $\rightarrow$ Ein sehr teures Delte, im Schnitt aber weiter O(1)
\end{itemize}

\subsubsection*{Zusatzaufgabe: Perfektes Hashing}


\end{document}