\documentclass[10pt,a4paper]{article}
\usepackage[utf8]{inputenc}
\usepackage[ngerman]{babel}
\usepackage{amsmath}
\usepackage{amsfonts}
\usepackage{amssymb}
\usepackage{graphicx}
\usepackage{hyperref}

\usepackage{float}

\usepackage{amsthm}
\usepackage{algorithm2e}

\usepackage{tikz}
\usepackage{cancel}
\usepackage{soul}

\usepackage{setspace}


\title{Altklausuren Antworten}
\date{}


\begin{document}
\maketitle
\href{https://github.com/ofenstichloch/Uni/blob/master/Kapitel\%20ADS/VL.pdf}{Landmesser Zusammenfassung}
\section*{Aufgabe 1}
\subsection*{O(log $n$) (amortisiert) für UNION und O(1) für FIND}

\begin{tabular}{ll}
     Feld name[1...n]:& name[x] = Name des Blocks der x enthält. 1 $\leq$ x $\leq$ n \\
     size[1..n]:& size[A] = Anzahl Elemente im Block A, initialisiert mit 1 \\ 
     L[1..n]: & L[A] = Liste aller Elemente in Block A, initialisiert L[i] = \{i\}
\end{tabular}
\begin{algorithm}[H]

    \textbf{Initialisierung:}	\\
	\Begin{
		\For{i := 1 to n}{
			name[i]=i \\
			size[i]=1 \\ 
			L[i]=\{i\}
		}
	}
	~\\
	\textbf{FIND(x):} \\
	\Begin{
		\textbf{return} name[x] \\
	}
	~\\
	\textbf{UNION(A,B):} \\
	\Begin{
		\uIf{size[A] $\leq$ size[B]}
		{
		    \ForEach{i in L[A]}{
		        name[i] = B \\ 
		    }
		    size[B] += size[A] \\ 
		    L[B] = L[B].concat(L[A]) \\ 
		}
		\Else{
    		\ForEach{i in L[B]}{
    		        name[i] = A \\ 
    		    }
    		    size[A] += size[B] \\ 
    		    L[A] = L[A].concat(L[B]) \\ 
    		}
	}
    
	
\end{algorithm}
\subsubsection*{Laufzeit:}
FIND(x): O(1) $\rightarrow$ Einfacher Zugriff auf ein Feld\\ 
UNION: O(log $n$) $\rightarrow$ x kann maximal log(n) mal seinen Namen ändern, da es sich nach jeder Namensänderung in einer doppelt so großen Menge befindet. 


\subsection*{O(1) für UNION und O(log $n$) für FIND}
\begin{tabular}{ll}
     Feld name[1...n]:& name[x] = Name des Blocks mit Wurzel x (hat nur Bedeutung, falls x Wurzel) \\
     Feld vater[1...n]:& vater[x] = $\begin{cases}
    Vater\ von\ x\ in\ seinem\ Baum\\
    0,\ falls\ x\ Wurzel
  \end{cases}$ \\
     Feld wurzel[1...n]:& wurzel[x] = Wurzel des Blocks mit Namen x \\
\end{tabular}
\begin{algorithm}[H]
	\textbf{Initialisierung:}	\\
	\Begin{
		\For{i := 1 to n}{
			vater[i]=0 \\
			name[i]=i \\
			wurzel[i]=i \\
		}
	}
	~\\
	\textbf{FIND(x):} \\
	\Begin{
		\While{vater[x] != 0}{
			x = vater[x]
		}
		\textbf{return} name[x] \\
	}
	~\\
	
	\textbf{UNION(A,B,C):} \\
	\Begin{
		$r_{1}$ = wurzel[A]	\\
		$r_{2}$ = wurzel[B]	\\
		\uIf{size[$r_{1}$] $\leq$ $r_{2}$}{
		    vater[$r_{1}$] = $r_{2}$	\\
    		name[$r_{2}$] = C	\\
    		wurzel[C] = $r_{2}$	\\ 
    		size[$r_{2}$] += size[$r_{1}$] \\ 
		}
		\Else{
		    vater[$r_{2}$] = $r_{1}$	\\
    		name[$r_{1}$] = C	\\
    		wurzel[C] = $r_{1}$	\\ 
    		size[$r_{1}$] += size[$r_{2}$] \\ 
		}
	}
	
\end{algorithm}

\subsubsection*{Laufzeit:}
FIND(x): O(log $n$) $\rightarrow$ Tiefe von x (max Höhe des entstehende Baums, n-1 möglich)\\ 
UNION: O(1) $\rightarrow$ Nur Pointer ändern


%%%%%%%%%%%%%%%%%%%%%%%%%
%		Aufgabe 2		%
%%%%%%%%%%%%%%%%%%%%%%%%%

\section*{Aufgabe 2}
\paragraph{Hashing mit Verkettung} löse Kollisionen nicht auf, speichere mehrere Schlüssel an der gleichen Position
\\

Speichere für jedes Ergebnis der Hashfunktion $h$ eine Liste

\textbf{Lookup(x)}: lineare Suche in Liste $T[h(x)]$ \\ 
\begin{itemize}
    \item Worst Case: alle Keys in derselben List $\rightarrow$ O(n)
    \item erwartete Zeit: O($\frac{n}{m}$)
    \item Belegungsfactor $\beta = \frac{n}{m}$ $\leftarrow$ erw. Länge einer Liste T[x]
    \item wenn $m \geq n$, d.h. $\beta \leq 1$ dann $\rightarrow$ erw. Laufzeit O(1)
\end{itemize}
        

\textbf{Insert(x)}: $x\notin S.$ Füge x an erst freie Stelle in $T[h(x)]$ ein

\textbf{Delete(x)}: Entferne x aus $T[h(x)]$

\begin{center}
	\resizebox{.6\columnwidth}{!}{
		

\tikzset{every picture/.style={line width=0.75pt}} %set default line width to 0.75pt        

\begin{tikzpicture}[x=0.75pt,y=0.75pt,yscale=-1,xscale=1][H]
%uncomment if require: \path (0,181); %set diagram left start at 0, and has height of 181

%Shape: Rectangle [id:dp31075993805350643] 
\draw   (20,20) -- (50,20) -- (50,40) -- (20,40) -- cycle ;
%Shape: Rectangle [id:dp07786860939393403] 
\draw   (20,60) -- (50,60) -- (50,80) -- (20,80) -- cycle ;
%Shape: Rectangle [id:dp4591623739856949] 
\draw   (20,40) -- (50,40) -- (50,60) -- (20,60) -- cycle ;
%Shape: Rectangle [id:dp983416511512617] 
\draw   (20,80) -- (50,80) -- (50,100) -- (20,100) -- cycle ;
%Shape: Rectangle [id:dp04528718387197084] 
\draw   (80,40) -- (110,40) -- (110,60) -- (80,60) -- cycle ;
%Shape: Rectangle [id:dp1270001423171485] 
\draw   (20,140) -- (50,140) -- (50,160) -- (20,160) -- cycle ;
%Shape: Rectangle [id:dp776903324067578] 
\draw   (20,120) -- (50,120) -- (50,140) -- (20,140) -- cycle ;
%Shape: Rectangle [id:dp08008497435086515] 
\draw   (20,100) -- (50,100) -- (50,120) -- (20,120) -- cycle ;
%Shape: Rectangle [id:dp6528086235651709] 
\draw   (110,40) -- (140,40) -- (140,60) -- (110,60) -- cycle ;
%Shape: Rectangle [id:dp6043880104015285] 
\draw   (170,40.5) -- (200,40.5) -- (200,60.5) -- (170,60.5) -- cycle ;
%Shape: Rectangle [id:dp5461726616378066] 
\draw   (200,40.5) -- (230,40.5) -- (230,60.5) -- (200,60.5) -- cycle ;
%Shape: Rectangle [id:dp31723847172342756] 
\draw   (80,100) -- (110,100) -- (110,120) -- (80,120) -- cycle ;
%Shape: Rectangle [id:dp9818861995587984] 
\draw   (110,100) -- (140,100) -- (140,120) -- (110,120) -- cycle ;
%Straight Lines [id:da5730569526468681] 
\draw    (35,50) -- (77,50) ;
\draw [shift={(80,50)}, rotate = 180] [fill={rgb, 255:red, 0; green, 0; blue, 0 }  ][line width=0.08]  [draw opacity=0] (8.93,-4.29) -- (0,0) -- (8.93,4.29) -- cycle    ;
\draw [shift={(35,50)}, rotate = 0] [color={rgb, 255:red, 0; green, 0; blue, 0 }  ][fill={rgb, 255:red, 0; green, 0; blue, 0 }  ][line width=0.75]      (0, 0) circle [x radius= 3.35, y radius= 3.35]   ;
%Straight Lines [id:da6922082890770056] 
\draw    (125,50) -- (167,50) ;
\draw [shift={(170,50)}, rotate = 180] [fill={rgb, 255:red, 0; green, 0; blue, 0 }  ][line width=0.08]  [draw opacity=0] (8.93,-4.29) -- (0,0) -- (8.93,4.29) -- cycle    ;
\draw [shift={(125,50)}, rotate = 0] [color={rgb, 255:red, 0; green, 0; blue, 0 }  ][fill={rgb, 255:red, 0; green, 0; blue, 0 }  ][line width=0.75]      (0, 0) circle [x radius= 3.35, y radius= 3.35]   ;
%Straight Lines [id:da7299557815790314] 
\draw    (35,110) -- (77,110) ;
\draw [shift={(80,110)}, rotate = 180] [fill={rgb, 255:red, 0; green, 0; blue, 0 }  ][line width=0.08]  [draw opacity=0] (8.93,-4.29) -- (0,0) -- (8.93,4.29) -- cycle    ;
\draw [shift={(35,110)}, rotate = 0] [color={rgb, 255:red, 0; green, 0; blue, 0 }  ][fill={rgb, 255:red, 0; green, 0; blue, 0 }  ][line width=0.75]      (0, 0) circle [x radius= 3.35, y radius= 3.35]   ;
%Straight Lines [id:da4088411111836039] 
\draw    (125,110) -- (160,110) -- (160,130) ;
\draw [shift={(160,130)}, rotate = 270] [color={rgb, 255:red, 0; green, 0; blue, 0 }  ][line width=0.75]    (0,5.59) -- (0,-5.59)   ;

%Straight Lines [id:da04469814123042082] 
\draw    (215,50) -- (250,50) -- (250,70) ;
\draw [shift={(250,70)}, rotate = 270] [color={rgb, 255:red, 0; green, 0; blue, 0 }  ][line width=0.75]    (0,5.59) -- (0,-5.59)   ;


% Text Node
\draw (13.5,50.5) node   [align=left] {2};
% Text Node
\draw (13.5,30.5) node   [align=left] {1};
% Text Node
\draw (13.5,70.5) node   [align=left] {3};
% Text Node
\draw (13.5,90.5) node   [align=left] {4};
% Text Node
\draw (13.5,110.5) node   [align=left] {5};
% Text Node
\draw (13.5,130.5) node   [align=left] {6};
% Text Node
\draw (13.5,149.5) node   [align=left] {7};
% Text Node
\draw (95,50) node   [align=left] {3};
% Text Node
\draw (185,50.5) node   [align=left] {17};
% Text Node
\draw (95,110) node   [align=left] {19};


\end{tikzpicture}

	}
\end{center}
meist wird als Hashfunktion einfaches Modulo verwendet.

\subsubsection*{Verbesserung Verdopplungs-Strategie:}
\begin{itemize}
    \item Immer wenn $\beta>2$, verdopple Tafelgröße $\rightarrow1$ sehr teures Insert (da alle Elemente mit neuer Hashfunktion umgespeichert werden), im Schnitt aber weiter O(1)\\ 
    \item Bei Delete und kleinem $\beta$: Tabelle kann kalbiert werden $\rightarrow$ Ein sehr teures Delte, im Schnitt aber weiter O(1)
\end{itemize}

\subsection*{Zusatzaufgabe: Perfektes Hashing}

%%%%%%%%%%%%%%%%%%%%%%%%%
%		Aufgabe 3		%
%%%%%%%%%%%%%%%%%%%%%%%%%
\section*{Aufgabe 3}

%%%%%%%%%%%%%%%%%%%%%%%%%
%		Aufgabe 4		%
%%%%%%%%%%%%%%%%%%%%%%%%%

\section*{Aufgabe 4}
	Sei G ein planarer Graph mit $ n \geq 3 $ Knoten und m Kanten, dann gilt $ m = 3n-6 $. dh $ m=O(n) $, also linear viele Kanten.
	\begin{proof}[H]
		Ein maximal planarer Graph ist ein planarer Graph, der durch Hinzufügen einer Kante $ (v,w) \notin E $ nicht-planar wird. Beobachtung: Alle Faces in jeder planaren Einbettung von G sind Dreiecke (Triangulierung). Jedes Face in einer Triangulierung hat 3 Rand-Kanten und jede Kante liegt am Rand von 3 Faces. 
		$$\Rightarrow 3f = 2m $$
		Einstetzen in Euler-Formel
		$$ n - m + \frac{2}{3}m = 2$$
		$$ m = 3n-6$$
		$ m \leq 3n-6 $ für beliebige planare Graphen
	\end{proof}

\subsection*{Zusatzaufgabe}
Sei $G$ ein \textbf{bipartiter} planarer Graph Dann gilt $ m \leq 2n-4 $. 

\paragraph{Beweis:} Keine Kreise ungerader Länge in bipartiten Graphen. Kleinstmögliche Fläche in einem bipartiten Graphen ist ein Viereck.$_{\qed}$


%%%%%%%%%%%%%%%%%%%%%%%%%
%		Aufgabe 5		%
%%%%%%%%%%%%%%%%%%%%%%%%%
\section*{Aufgabe 5}
\paragraph{Split-Find-Problem}

Idee - Unkehrung von Union-Find
Feld name[1,...,n]

\begin{algorithm}[H]
	\textbf{Initialisierung:} \\
	\Begin{
		\For{$\forall i, 1\leq i \leq n$}{
			name[i] = 1
		}
	}
	~\\
	\textbf{Find(i)}
	\Begin{
		return name[i] \hspace{.5cm} $\rightarrow\mathcal{O}(1)$
	}
	\textbf{Split(i)}
	\begin{itemize}
		\item Neuer Name des neuen Intervalls das durch Split entsteht ++count
		\item Relabel the smaller half
		\item Laufe parallel d.h. abwechselnd nach links und rechts von $i$ aus, bis Intervallgrenze erreicht d.h. $name[i]\neq name[betrachtetes Element]$
		\item Nenne den Teil um, der kleiner ist $[a,i]$ oder $[i+1,b]$ indem nochmals über diesen Teil gelaufen wird. name[betrachtetes Element] = count
	\end{itemize}
\end{algorithm}
	\begin{center}
		\resizebox{.5\columnwidth}{!}{
			\tikzset{every picture/.style={line width=0.75pt}} %set default line width to 0.75pt        
\begin{tikzpicture}[x=0.75pt,y=0.75pt,yscale=-1,xscale=1]
%uncomment if require: \path (0,74.33332824707031); %set diagram left start at 0, and has height of 74.33332824707031

%Shape: Rectangle [id:dp6287463531133186] 
\draw   (20,20) -- (50,20) -- (50,40) -- (20,40) -- cycle ;
%Shape: Rectangle [id:dp6712332802681875] 
\draw   (50,20) -- (80,20) -- (80,40) -- (50,40) -- cycle ;
%Shape: Rectangle [id:dp8581103754160746] 
\draw   (80,20) -- (110,20) -- (110,40) -- (80,40) -- cycle ;
%Shape: Rectangle [id:dp6750047382984976] 
\draw   (110,20) -- (140,20) -- (140,40) -- (110,40) -- cycle ;
%Shape: Rectangle [id:dp7403561919997843] 
\draw   (140,20) -- (170,20) -- (170,40) -- (140,40) -- cycle ;
%Shape: Rectangle [id:dp644418423147074] 
\draw   (170,20) -- (200,20) -- (200,40) -- (170,40) -- cycle ;
%Shape: Rectangle [id:dp6636672838250548] 
\draw   (200,20) -- (230,20) -- (230,40) -- (200,40) -- cycle ;
%Shape: Rectangle [id:dp2893216606013862] 
\draw   (80,40) -- (110,40) -- (110,60) -- (80,60) -- cycle ;
%Shape: Rectangle [id:dp4354422116955303] 
\draw   (20,40) -- (50,40) -- (50,60) -- (20,60) -- cycle ;
%Shape: Rectangle [id:dp9120747157270248] 
\draw   (50,40) -- (80,40) -- (80,60) -- (50,60) -- cycle ;

% Text Node
\draw (35,30) node   [align=left] {\cancel{1}};
% Text Node
\draw (65,30) node   [align=left] {\cancel{1}};
% Text Node
\draw (95,30) node   [align=left] {\cancel{1}};
% Text Node
\draw (155,30) node   [align=left] {1};
% Text Node
\draw (125,30) node   [align=left] {1};
% Text Node
\draw (185,30) node   [align=left] {1};
% Text Node
\draw (215,30) node   [align=left] {1};
% Text Node
\draw (35,50) node   [align=left] {2};
% Text Node
\draw (65,50) node   [align=left] {2};
% Text Node
\draw (95,50) node   [align=left] {2};
% Text Node
\draw (16.5,10.5) node   [align=left] {a};
% Text Node
\draw (233.5,10.5) node   [align=left] {b};
% Text Node
\draw (113.5,9.5) node   [align=left] {i};


\end{tikzpicture}

		}
	\end{center}
	$\Rightarrow$ Kosten für 1 Split $\mathcal{O}(2*Laenge des kuerzeren Intervalls)$ \\
	\textbf{Analyse} $\mathcal{O}(\#Namensaenderungen): max\frac{\text{Laenge des umzubennenendes Intervall}}{2}$ $\Rightarrow\mathcal{O}(\log n)$



%%%%%%%%%%%%%%%%%%%%%%%%%
%		Aufgabe 6		%
%%%%%%%%%%%%%%%%%%%%%%%%%

\section*{Aufgabe 6}
\paragraph{Algorithmus von Dijkstra:}~\\
\begin{algorithm}[H]
\Begin{
	\ForEach{$v\in V$}{
		DIST[v]$\leftarrow\infty$ \\
		PRED[v]$\leftarrow$NULL
	}
	DIST[s]$\leftarrow$0 \\
	PQ.insert(v,0) \\
	\While{not PQ.empty()}{
		u $\leftarrow$ PQ.delmin() //liefertInfo \\
		\ForEach{$v\in V$ mit $(u,v)\in E$}{
			d$\leftarrow$DIST[u]+c(u,v) \\
			\If{d<DIST[v]}{
				\If{DIST[v]==$\infty$}{PQ.insert(v,d)}
				\Else{PQ.decrease(v,d)}
				DIST[v]$\leftarrow$d \\
				PRED[u]$\leftarrow$u \\
			}
		}		
	}
}
	
\end{algorithm}

\paragraph{Laufzeitanalyse:}
	$\mathcal{O}(\Sigma_{v\in V}(1+outdeg(v))+PQ_Operationen)n*(T_{insert}+T_{delmin}+T_{empty})+m*T_{decrease}$
	\begin{itemize}
		\item Bei Binärem Heap: $\mathcal{O}(n*log(n)+m*log(n)=\mathcal{O}((n+m)*log(n)$
		\item Fibonacci-Heap:Amortisierte Analyse ist ok, da Gesamtlaufzeit betrachtet. $\mathcal{O}(n*log(n)+m)$, insert+empty = $\mathcal{O}(1)$, delmin=$\mathcal{O}(log(n))$, decrease = $\mathcal{O}(1)$
	\end{itemize}

%%%%%%%%%%%%%%%%%%%%%%%%%
%		Aufgabe 7		%
%%%%%%%%%%%%%%%%%%%%%%%%%
\section*{Aufgabe 7}

Eine Planare Einbettung ist genau dann eindeutig wenn diese 3-fach zusammenhängend ist:

\begin{center}
	\resizebox{.2\columnwidth}{!}{
		

\tikzset{every picture/.style={line width=0.75pt}} %set default line width to 0.75pt        

\begin{tikzpicture}[x=0.75pt,y=0.75pt,yscale=-1,xscale=1][H]
%uncomment if require: \path (0,75.67707824707031); %set diagram left start at 0, and has height of 75.67707824707031

%Straight Lines [id:da30194207921776495] 
\draw    (30,20) -- (80,20) ;
\draw [shift={(80,20)}, rotate = 0] [color={rgb, 255:red, 0; green, 0; blue, 0 }  ][fill={rgb, 255:red, 0; green, 0; blue, 0 }  ][line width=0.75]      (0, 0) circle [x radius= 3.35, y radius= 3.35]   ;
\draw [shift={(30,20)}, rotate = 0] [color={rgb, 255:red, 0; green, 0; blue, 0 }  ][fill={rgb, 255:red, 0; green, 0; blue, 0 }  ][line width=0.75]      (0, 0) circle [x radius= 3.35, y radius= 3.35]   ;
%Straight Lines [id:da7970046183091499] 
\draw    (30,60) -- (80,60) ;
\draw [shift={(80,60)}, rotate = 0] [color={rgb, 255:red, 0; green, 0; blue, 0 }  ][fill={rgb, 255:red, 0; green, 0; blue, 0 }  ][line width=0.75]      (0, 0) circle [x radius= 3.35, y radius= 3.35]   ;
\draw [shift={(30,60)}, rotate = 0] [color={rgb, 255:red, 0; green, 0; blue, 0 }  ][fill={rgb, 255:red, 0; green, 0; blue, 0 }  ][line width=0.75]      (0, 0) circle [x radius= 3.35, y radius= 3.35]   ;
%Straight Lines [id:da7704668417599982] 
\draw    (30,20) -- (30,42.68) -- (30,60) ;
\draw [shift={(30,60)}, rotate = 90] [color={rgb, 255:red, 0; green, 0; blue, 0 }  ][fill={rgb, 255:red, 0; green, 0; blue, 0 }  ][line width=0.75]      (0, 0) circle [x radius= 3.35, y radius= 3.35]   ;
\draw [shift={(30,20)}, rotate = 90] [color={rgb, 255:red, 0; green, 0; blue, 0 }  ][fill={rgb, 255:red, 0; green, 0; blue, 0 }  ][line width=0.75]      (0, 0) circle [x radius= 3.35, y radius= 3.35]   ;
%Straight Lines [id:da22878533995815276] 
\draw    (80,20) -- (80,42.68) -- (80,60) ;
\draw [shift={(80,60)}, rotate = 90] [color={rgb, 255:red, 0; green, 0; blue, 0 }  ][fill={rgb, 255:red, 0; green, 0; blue, 0 }  ][line width=0.75]      (0, 0) circle [x radius= 3.35, y radius= 3.35]   ;
\draw [shift={(80,20)}, rotate = 90] [color={rgb, 255:red, 0; green, 0; blue, 0 }  ][fill={rgb, 255:red, 0; green, 0; blue, 0 }  ][line width=0.75]      (0, 0) circle [x radius= 3.35, y radius= 3.35]   ;




\end{tikzpicture}

	}
\end{center}

Gleicher Graph verschiedene Einbettungen:

\begin{center}
	\resizebox{.5\columnwidth}{!}{
		

\tikzset{every picture/.style={line width=0.75pt}} %set default line width to 0.75pt        

\begin{tikzpicture}[x=0.75pt,y=0.75pt,yscale=-1,xscale=1][H]
%uncomment if require: \path (0,132.34375); %set diagram left start at 0, and has height of 132.34375

%Straight Lines [id:da19870838641021127] 
\draw    (60,20) -- (100,60) ;
\draw [shift={(100,60)}, rotate = 45] [color={rgb, 255:red, 0; green, 0; blue, 0 }  ][fill={rgb, 255:red, 0; green, 0; blue, 0 }  ][line width=0.75]      (0, 0) circle [x radius= 3.35, y radius= 3.35]   ;
\draw [shift={(60,20)}, rotate = 45] [color={rgb, 255:red, 0; green, 0; blue, 0 }  ][fill={rgb, 255:red, 0; green, 0; blue, 0 }  ][line width=0.75]      (0, 0) circle [x radius= 3.35, y radius= 3.35]   ;
%Straight Lines [id:da33901297706652644] 
\draw    (100,60) -- (60,100) ;
\draw [shift={(60,100)}, rotate = 135] [color={rgb, 255:red, 0; green, 0; blue, 0 }  ][fill={rgb, 255:red, 0; green, 0; blue, 0 }  ][line width=0.75]      (0, 0) circle [x radius= 3.35, y radius= 3.35]   ;
\draw [shift={(100,60)}, rotate = 135] [color={rgb, 255:red, 0; green, 0; blue, 0 }  ][fill={rgb, 255:red, 0; green, 0; blue, 0 }  ][line width=0.75]      (0, 0) circle [x radius= 3.35, y radius= 3.35]   ;
%Straight Lines [id:da2046055571063723] 
\draw    (60,20) -- (20,60) ;
\draw [shift={(20,60)}, rotate = 135] [color={rgb, 255:red, 0; green, 0; blue, 0 }  ][fill={rgb, 255:red, 0; green, 0; blue, 0 }  ][line width=0.75]      (0, 0) circle [x radius= 3.35, y radius= 3.35]   ;
\draw [shift={(60,20)}, rotate = 135] [color={rgb, 255:red, 0; green, 0; blue, 0 }  ][fill={rgb, 255:red, 0; green, 0; blue, 0 }  ][line width=0.75]      (0, 0) circle [x radius= 3.35, y radius= 3.35]   ;
%Straight Lines [id:da8631533342403717] 
\draw    (20,60) -- (60,100) ;
\draw [shift={(60,100)}, rotate = 45] [color={rgb, 255:red, 0; green, 0; blue, 0 }  ][fill={rgb, 255:red, 0; green, 0; blue, 0 }  ][line width=0.75]      (0, 0) circle [x radius= 3.35, y radius= 3.35]   ;
\draw [shift={(20,60)}, rotate = 45] [color={rgb, 255:red, 0; green, 0; blue, 0 }  ][fill={rgb, 255:red, 0; green, 0; blue, 0 }  ][line width=0.75]      (0, 0) circle [x radius= 3.35, y radius= 3.35]   ;
%Straight Lines [id:da5710000824963755] 
\draw    (60,20) -- (60,60) ;
\draw [shift={(60,60)}, rotate = 90] [color={rgb, 255:red, 0; green, 0; blue, 0 }  ][fill={rgb, 255:red, 0; green, 0; blue, 0 }  ][line width=0.75]      (0, 0) circle [x radius= 3.35, y radius= 3.35]   ;
\draw [shift={(60,20)}, rotate = 90] [color={rgb, 255:red, 0; green, 0; blue, 0 }  ][fill={rgb, 255:red, 0; green, 0; blue, 0 }  ][line width=0.75]      (0, 0) circle [x radius= 3.35, y radius= 3.35]   ;
%Straight Lines [id:da8325932446920639] 
\draw    (60,60) -- (60,100) ;
\draw [shift={(60,100)}, rotate = 90] [color={rgb, 255:red, 0; green, 0; blue, 0 }  ][fill={rgb, 255:red, 0; green, 0; blue, 0 }  ][line width=0.75]      (0, 0) circle [x radius= 3.35, y radius= 3.35]   ;
\draw [shift={(60,60)}, rotate = 90] [color={rgb, 255:red, 0; green, 0; blue, 0 }  ][fill={rgb, 255:red, 0; green, 0; blue, 0 }  ][line width=0.75]      (0, 0) circle [x radius= 3.35, y radius= 3.35]   ;
%Straight Lines [id:da8472289159892161] 
\draw    (60,60) -- (100,60) ;
\draw [shift={(100,60)}, rotate = 0] [color={rgb, 255:red, 0; green, 0; blue, 0 }  ][fill={rgb, 255:red, 0; green, 0; blue, 0 }  ][line width=0.75]      (0, 0) circle [x radius= 3.35, y radius= 3.35]   ;
\draw [shift={(60,60)}, rotate = 0] [color={rgb, 255:red, 0; green, 0; blue, 0 }  ][fill={rgb, 255:red, 0; green, 0; blue, 0 }  ][line width=0.75]      (0, 0) circle [x radius= 3.35, y radius= 3.35]   ;
%Straight Lines [id:da5992638659689509] 
\draw    (200,21) -- (240,61) ;
\draw [shift={(240,61)}, rotate = 45] [color={rgb, 255:red, 0; green, 0; blue, 0 }  ][fill={rgb, 255:red, 0; green, 0; blue, 0 }  ][line width=0.75]      (0, 0) circle [x radius= 3.35, y radius= 3.35]   ;
\draw [shift={(200,21)}, rotate = 45] [color={rgb, 255:red, 0; green, 0; blue, 0 }  ][fill={rgb, 255:red, 0; green, 0; blue, 0 }  ][line width=0.75]      (0, 0) circle [x radius= 3.35, y radius= 3.35]   ;
%Straight Lines [id:da9374481511290493] 
\draw    (240,61) -- (200,101) ;
\draw [shift={(200,101)}, rotate = 135] [color={rgb, 255:red, 0; green, 0; blue, 0 }  ][fill={rgb, 255:red, 0; green, 0; blue, 0 }  ][line width=0.75]      (0, 0) circle [x radius= 3.35, y radius= 3.35]   ;
\draw [shift={(240,61)}, rotate = 135] [color={rgb, 255:red, 0; green, 0; blue, 0 }  ][fill={rgb, 255:red, 0; green, 0; blue, 0 }  ][line width=0.75]      (0, 0) circle [x radius= 3.35, y radius= 3.35]   ;
%Straight Lines [id:da3321652578255401] 
\draw    (200,21) -- (160,61) ;
\draw [shift={(160,61)}, rotate = 135] [color={rgb, 255:red, 0; green, 0; blue, 0 }  ][fill={rgb, 255:red, 0; green, 0; blue, 0 }  ][line width=0.75]      (0, 0) circle [x radius= 3.35, y radius= 3.35]   ;
\draw [shift={(200,21)}, rotate = 135] [color={rgb, 255:red, 0; green, 0; blue, 0 }  ][fill={rgb, 255:red, 0; green, 0; blue, 0 }  ][line width=0.75]      (0, 0) circle [x radius= 3.35, y radius= 3.35]   ;
%Straight Lines [id:da38066349715363756] 
\draw    (160,61) -- (200,101) ;
\draw [shift={(200,101)}, rotate = 45] [color={rgb, 255:red, 0; green, 0; blue, 0 }  ][fill={rgb, 255:red, 0; green, 0; blue, 0 }  ][line width=0.75]      (0, 0) circle [x radius= 3.35, y radius= 3.35]   ;
\draw [shift={(160,61)}, rotate = 45] [color={rgb, 255:red, 0; green, 0; blue, 0 }  ][fill={rgb, 255:red, 0; green, 0; blue, 0 }  ][line width=0.75]      (0, 0) circle [x radius= 3.35, y radius= 3.35]   ;
%Straight Lines [id:da5420722657559385] 
\draw    (200,21) -- (200,61) ;
\draw [shift={(200,61)}, rotate = 90] [color={rgb, 255:red, 0; green, 0; blue, 0 }  ][fill={rgb, 255:red, 0; green, 0; blue, 0 }  ][line width=0.75]      (0, 0) circle [x radius= 3.35, y radius= 3.35]   ;
\draw [shift={(200,21)}, rotate = 90] [color={rgb, 255:red, 0; green, 0; blue, 0 }  ][fill={rgb, 255:red, 0; green, 0; blue, 0 }  ][line width=0.75]      (0, 0) circle [x radius= 3.35, y radius= 3.35]   ;
%Straight Lines [id:da14578509110975646] 
\draw    (200,61) -- (200,101) ;
\draw [shift={(200,101)}, rotate = 90] [color={rgb, 255:red, 0; green, 0; blue, 0 }  ][fill={rgb, 255:red, 0; green, 0; blue, 0 }  ][line width=0.75]      (0, 0) circle [x radius= 3.35, y radius= 3.35]   ;
\draw [shift={(200,61)}, rotate = 90] [color={rgb, 255:red, 0; green, 0; blue, 0 }  ][fill={rgb, 255:red, 0; green, 0; blue, 0 }  ][line width=0.75]      (0, 0) circle [x radius= 3.35, y radius= 3.35]   ;
%Straight Lines [id:da08019819679875595] 
\draw    (200,61) -- (240,61) ;
\draw [shift={(240,61)}, rotate = 0] [color={rgb, 255:red, 0; green, 0; blue, 0 }  ][fill={rgb, 255:red, 0; green, 0; blue, 0 }  ][line width=0.75]      (0, 0) circle [x radius= 3.35, y radius= 3.35]   ;
\draw [shift={(200,61)}, rotate = 0] [color={rgb, 255:red, 0; green, 0; blue, 0 }  ][fill={rgb, 255:red, 0; green, 0; blue, 0 }  ][line width=0.75]      (0, 0) circle [x radius= 3.35, y radius= 3.35]   ;

% Text Node
\draw (13.5,59.5) node   [align=left] {1};
% Text Node
\draw (53.5,9.5) node   [align=left] {2};
% Text Node
\draw (113.5,59.5) node   [align=left] {3};
% Text Node
\draw (63.5,110.5) node   [align=left] {4};
% Text Node
\draw (46.5,60.5) node   [align=left] {5};
% Text Node
\draw (153.5,60.5) node   [align=left] {1};
% Text Node
\draw (193.5,10.5) node   [align=left] {2};
% Text Node
\draw (253.5,60.5) node   [align=left] {5};
% Text Node
\draw (203.5,111.5) node   [align=left] {4};
% Text Node
\draw (186.5,61.5) node   [align=left] {3};


\end{tikzpicture}

	}
\end{center}


%%%%%%%%%%%%%%%%%%%%%%%%%
%		Aufgabe 8		%
%%%%%%%%%%%%%%%%%%%%%%%%%
\section*{Aufgabe 8}

\end{document}