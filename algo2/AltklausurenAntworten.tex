\documentclass[10pt,a4paper]{article}
\usepackage[utf8]{inputenc}
\usepackage[ngerman]{babel}
\usepackage{amsmath}
\usepackage{amsfonts}
\usepackage{amssymb}
\usepackage{graphicx}
\usepackage{hyperref}

\usepackage{amsthm}
\usepackage{algorithm2e}

\usepackage{setspace}


\title{Altklausuren Antworten}
\date{}


\begin{document}
\maketitle
\href{https://github.com/ofenstichloch/Uni/blob/master/Kapitel\%20ADS/VL.pdf}{Landmesser Zusammenfassung}
\section*{Aufgabe 1}
\subsection*{O(log $n$) (amortisiert) für UNION und O(1) für FIND}
\subsection*{O(1) für UNION und O(log$n$) für FIND}
\begin{algorithm}
	\textbf{Initialisierung:}	\\
	\Begin{
		\For{i := 1 to n}{
			vater[i]=0 \\
			name[i]=i \\
			wurzel[i]=i \\
		}
	}
	~\\
	\textbf{FIND(x):} \\
	\Begin{
		\While{vater[x] != 0}{
			x = vater[x]
		}
		\textbf{return} name[x] \\
	}
	~\\
	
	\textbf{UNION(A,B,C):} \\
	\Begin{
		$r_{1}$ = wurzel[A]	\\
		$r_{2}$ = wurzel[B]	\\
		vater[$r_{1}$] = $r_{2}$	\\
		name[$r_{2}$] = C	\\
		wurzel[C] = $r_{2}$	\\
	}
	
\end{algorithm}

%%%%%%%%%%%%%%%%%%%%%%%%%
%		Aufgabe 2		%
%%%%%%%%%%%%%%%%%%%%%%%%%

\section*{Aufgabe 2}
\subsection*{Zusatzaufgabe: Perfektes Hashing}

%%%%%%%%%%%%%%%%%%%%%%%%%
%		Aufgabe 3		%
%%%%%%%%%%%%%%%%%%%%%%%%%
\section*{Aufgabe3}

%%%%%%%%%%%%%%%%%%%%%%%%%
%		Aufgabe 4		%
%%%%%%%%%%%%%%%%%%%%%%%%%

\section*{Aufgabe 4}
	Sei G ein planarer Graph mit $ n \geq 3 $ Knoten und m Kanten, dann gilt $ m = 3n-6 $. dh $ m=O(n) $, also linear viele Kanten.
	\begin{proof}
		Ein maximal planarer Graph ist ein planarer Graph, der durch Hinzufügen einer Kante $ (v,w) \notin E $ nicht-planar wird. Beobachtung: Alle Faces in jeder planaren Einbettung von G sind Dreiecke (Triangulierung). Jedes Face in einer Triangulierung hat 3 Rand-Kanten und jede Kante liegt am Rand von 3 Faces. 
		$$\Rightarrow 3f = 2m $$
		Einstetzen in Euler-Formel
		$$ n - m + \frac{2}{3}m = 2$$
		$$ m = 3n-6$$
		$ m \leq 3n-6 $ für beliebige planare Graphen
	\end{proof}

\subsection*{Zusatzaufgabe}
Sei $G$ ein \textbf{bipartiter} planarer Graph Dann gilt $ m \leq 2n-4 $. 

\paragraph{Beweis:} Keine Kreise ungerader Länge in bipartiten Graphen. Kleinstmögliche Fläche in einem bipartiten Graphen ist ein Viereck.


%%%%%%%%%%%%%%%%%%%%%%%%%
%		Aufgabe 5		%
%%%%%%%%%%%%%%%%%%%%%%%%%
\section*{Aufgabe 5}

%%%%%%%%%%%%%%%%%%%%%%%%%
%		Aufgabe 6		%
%%%%%%%%%%%%%%%%%%%%%%%%%

\section*{Aufgabe 6}
\paragraph{Algorithmus von Dijkstra:}~
\begin{algorithm}
	\ForEach{$v\in V$}{
		DIST[v]$\leftarrow\infty$ \\
		PRED[v]$\leftarrow$NULL
	}
	DIST[s]$\leftarrow$0 \\
	PQ.insert(v,0) \\
	\While{not PQ.empty()}{
		u $\leftarrow$ PQ.delmin() //liefertInfo \\
		\ForEach{$v\in V$ mit $(u,v)\in E$}{
			d$\leftarrow$DIST[u]+c(u,v) \\
			\If{d<DIST[v]}{
				\If{DIST[v]==$\infty$}{PQ.insert(v,d)}
				\Else{PQ.decrease(v,d)}
				DIST[v]$\leftarrow$d \\
				PRED[u]$\leftarrow$u \\
			}
		}		
	}
\end{algorithm}

\paragraph{Laufzeitanalyse:}
	$\mathcal{O}(\Sigma_{v\in V}(1+outdeg(v))+PQ_Operationen)n*(T_{insert}+T_{delmin}+T_{empty})+m*T_{decrease}$
	\begin{itemize}
		\item Bei Binärem Heap: $\mathcal{O}(n*log(n)+m*log(n)=\mathcal{O}((n+m)*log(n)$
		\item Fibonacci-Heap:Amortisierte Analyse ist ok, da Gesamtlaufzeit betrachtet. $\mathcal{O}(n*log(n)+m)$, insert+empty = $\mathcal{O}(1)$, delmin=$\mathcal{O}(log(n))$, decrease = $\mathcal{O}(1)$
	\end{itemize}

%%%%%%%%%%%%%%%%%%%%%%%%%
%		Aufgabe 7		%
%%%%%%%%%%%%%%%%%%%%%%%%%
\section*{Aufgabe 7}


%%%%%%%%%%%%%%%%%%%%%%%%%
%		Aufgabe 8		%
%%%%%%%%%%%%%%%%%%%%%%%%%
\section*{Aufgabe 8}

\end{document}