\documentclass[11pt,a4paper,parskip=half ]{scrartcl}
\usepackage[utf8]{inputenc}
\usepackage[ngerman]{babel}
\usepackage{amsmath}
\usepackage{amsfonts}
\usepackage{amssymb}
\usepackage{graphicx}
\usepackage{xcolor}
\usepackage{float}
\usepackage{graphicx}
\graphicspath{./}
\usepackage{hyperref}

\author{Aaron Winziers - 1176638}
\title{Einführung in die Computergrafik WS 2018\\\LARGE{Übungsblatt 4}}

\begin{document}
	\maketitle
	
	\section*{Aufgabe 1}
	\begin{center}
		$
		R = 
		\begin{bmatrix}
		\cos(\frac{\pi}{2}) & -\sin(\frac{\pi}{2}) & 0	\\
		\sin(\frac{\pi}{2}) & \cos(\frac{\pi}{2}) & 0	\\
		0 & 0 & 1	\\
		\end{bmatrix} = 
		\begin{bmatrix}
		0 & -1 & 0	\\
		1 & 0 & 0	\\
		0 & 0 & 1	\\
		\end{bmatrix}		
		S = 
		\begin{bmatrix}
		2 & 0 & 0	\\
		0 & 3 & 0	\\
		0 & 0 & 1	\\
		\end{bmatrix}
		R\times S = 
		\begin{bmatrix}
		0 & -3 & 0	\\
		2 & 0 & 0	\\
		0 & 0 & 1	\\
		\end{bmatrix}
		S\times R = 
		\begin{bmatrix}
		0 & -2 & 0	\\
		3 & 0 & 0	\\
		0 & 0 & 1	\\
		\end{bmatrix}
		$
	\end{center}
	\textit{aufgabe1.html} beinhaltet die Aufgabe, und \textit{aufgabe1überprüfen.svg} die Prüfung mit translate, rotate und scale.
	
	\section*{Aufgabe 2}
	Siehe \textit{aufgabe2.html}
	
\end{document}