\documentclass[11pt,a4paper,parskip=quarter ]{article}
\usepackage[utf8]{inputenc}
\usepackage[ngerman]{babel}
\usepackage{amsmath}
\usepackage{amsfonts}
\usepackage{amssymb}
\usepackage{graphicx}
\usepackage{xcolor}
\usepackage{float}
\usepackage{graphicx}
\usepackage{hyperref}
\usepackage{tikz}
 
\usepackage{fancyhdr}

\pagestyle{fancy}
\rhead{Aaron Winziers - 1176638}

\author{Aaron Winziers - 1176638}
\title{Einführung in die Computergrafik SS 2019\\\Large{Übungsblatt 6}}

\begin{document}
	\maketitle
	\section*{Aufgabe 1}
	\subsection*{a)}(A, F, B, C, A, E, F, D, C, E)
	\subsection*{b)}(F, A, B, C, D, E, A),(A, B, C, D, E)
	
	\section*{Aufgabe 2}
	\subsection*{a)}
	
	Sei $n$ die Anzahl der Kanten und $V_{n} = \{v_{0}, ... , v_{n-1}\}$ die Menge der Knoten im Polygon und seien die Knoten gegen den Uhrzeigersinn durchnummeriert. Dann ist $A(n)=(n-2)\ast\pi$ 
	
	\paragraph*{Induktionsanfang}
	
	Sei $n=3$. Dann ist $A(n)=A(3)=(3-2)\ast\pi=\pi\checkmark$
	
	\paragraph{Induktionsschritt}
	
	$A(n+1)$. Bilde ein Dreieck mit Knoten $v_{0},v_{n-2},v_{n-1}$ und ein weiteres Polygon mit Knoten $v_{0},...,v_{n-2}$(in Abbildung 1 an einem Beispiel mit $n=5$ verdeutlicht), dann ist
	\begin{align*}
		A(n+1)	&=A(n)+A(3)	\\
				&=(n-2)\ast\pi+\pi
	\end{align*}
	
	\begin{center}
		\begin{figure}
			\centering
			\begin{tikzpicture}
			\path[draw] (0,0) node[left] (6) {$v_{0}$}
			-- (1.5,-1) node[below] (5) {$v_{1}$}
			-- (3.5, -0.8) node[below] (4) {$v_{2}$}
			-- (5,0) node[right] (3) {$v_{3}$}
			-- (4, 1.5) node[above] (2) {$v_{4}$}
			-- (1,2) node[above] (1) {$v_{5}$}
			-- cycle
			-- (4,1.5);
			\end{tikzpicture}
			\caption{Beispiel}
		\end{figure}
	\end{center}

\subsection*{b)}
	Da jeder Außenwinkel durch $\pi-Innenwinkel$ berechnet werden kann und $A(n)$ die Summe aller Innenwinkel berechnet:	\\
	
	Sei $n$ die Anzahl der Kanten in einem Polygon
	\begin{align*}
		\Sigma(Aussenwinkel)&=(\pi\ast n)-A(n) 		\\
							&=\pi n - (n-2)\ast\pi 	\\
							&=\pi n-(\pi n -2\pi) 	\\
							&=2\pi					\\
	\end{align*}
	
	
%\section*{Aufgabe 3}
%	Sei $P$ ein Polygon mit $h$ Löcher und insgesamt $n$ Ecken, inklusive die Ecken der Löcher.
%	
%	Seien $n_0$ die Anzahl der Ecken in der Außenkante von $P$, und $n_i$ die Anzahl der Ecken des $i-ten$ Löchs. Wie in Aufgabe 2a bewiesen ist die Summe der inneren Winkel des Polygons gleich $(n_0-2)*\pi$, und hiermit kann man auch die Summe der GESAMTaußenwinkel herleiten:
%	\begin{align*}
%	\Sigma(Aussenwinkel)&=2\pi n- (n-2)\ast\pi		\\
%	&=2\pi n -(\pi n -2\pi)1 	\\
%	&=2\pi n-\pi n+2\pi 	\\
%	&=\pi n+2\pi					\\
%	&=\pi(n+2)
%	\end{align*}
%	
%	Sei $t$ die Anzahl der benötigten Dreiecke, dann
%	\begin{align*}
%		180[(n_0-2)+(n_1+2)+...(n_i+2)] = 180t
%	\end{align*}
%	oder
%	\begin{align*}
%		t=n+2h-2
%	\end{align*}
\end{document}