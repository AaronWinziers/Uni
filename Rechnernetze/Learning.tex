\documentclass[10pt,a4paper]{article}
\usepackage[utf8]{inputenc}
\usepackage{amsmath}
\usepackage{amsfonts}
\usepackage{amssymb}
\usepackage{graphicx}
\author{Aaron Winziers}
\title{Rechnernetze}
\begin{document}
	\maketitle
	\tableofcontents
\section{Einführung}
	\subsection{Addressen}
		4 parts: 
			\begin{itemize}
				\item Protokoll
				\item Host/Host IP
				\item Port
				\item Resource
			\end{itemize}
	\subsection{Protokolle}	
		\begin{itemize}
			\item dns
			\item http/https
			\item ssh
			\item smtp
			\item etc...
		\end{itemize}
	
	
\section{ISO-Modell}
	\subsection{Layers:}
		\begin{enumerate}
			\item Physical Layer - Definition der (elektrischen, optischen) Schnittstellen
			\subitem PCI, SCSI, USB
			\item Link Layer - Fehlerfreier Kanal zwischen 2 Knoten
			\subitem Ethernet
			\item Network Layer - Host-to-Host Protokoll
			\subitem 2 Teile: Innerhalb eines Netzes, Zwischen Netzen
			\subitem IPv4, IPv6
			\item Transport Layer - Ende-zu-Ende Protokoll
			\subitem TCP, UDP
			\item Session Layer - Kommunikationssitzung
			\item Presentation Layer - Umwandlung von Datenformen
			\item Application Layer - Spezifische Anwendungsprotokolle 
		\end{enumerate}
	\subsection{Router/Gateway}
		\begin{itemize}
			\item Router - leitet nur weiter, kommt nur bis Ebene 3
			\item Gateway - kann auch filtern, get auf Ebene 7 hoch
		\end{itemize}
	
	
\section{Topologien}
	\subsection{Typische Topologien}
		\begin{itemize}
			\item Bus
			\item Stern
			\item Hypercube
			\item Gatter
			\item Torus/Donut
			\item Butterfly(Flaches Hypercube)
		\end{itemize}
	\subsection{Busse}
		Kollisionsverhinderung:
			\begin{itemize}
				\item FDMA - Frequency Divison Multiple Access
				\item TDMA - Time Division Multiple Access
				\item CDMA - Code Division Multiple Access
				\item CSMA - Listen before Talk
			\end{itemize}


\section{DNS}
	\subsection{Domain Name Service}
		Host.Subdomain. ... .Subdomain.TLD	\\
		TLD - Top Level Domain
		
\section{Ethernet??}

\section{IP und ICMP}
	\subsection{All Pairs Shortest Paths}
		Floyd-Warshall-Algorithmus $O(n^{3})$
		
	\subsection{IP}
		\begin{itemize}
			\item Auf Ebene 3 im OSI-Modell
			\item "3" Klassen von Netzwerken
			\begin{itemize}
				\item Class A : 0|netID(7)|hostID(24)
				\item Class B : 10|netID(14)|hostID(16)
				\item Class C : 110|netID(21)|hostID(8)
			\end{itemize}
			\item ARP \& RARP (Address Resolution Protocol \& Reverse Address Resolution Protocol)
		\end{itemize}
	\subsection{ICMP}
		Internet Control Message Protocol - Erledigt administrative Aufgaben im Internet - Fehler-/Informationsmeldungen 
	\subsection{Super-/Subnetting}
		Nehme Host-Anteil und  Verwende Local-Routing um weitere Subnetze zu erzeugen
		\begin{center}
			10|netID(14)|subnetID(8)|hostID(8)
		\end{center}
		CIDR (Classless Inter-Domain Routing) - Keine Netzwerkklassen mehr	\\
		NAT (Network Address Translation)
		
\section{TCP \& UDP}
	\subsection{UDP}
		User Datagram Protocol
	\begin{itemize}
		\item ACK/NACK (impliziter vs expliziter Quittungsbetrieb)
		\item Maximierung des Speichermediums
		\item 3-Way Handshake
		\subitem Client -> SYN, ISN Client -> Server
		\subitem Client <- SYN, ISN Server, ACK ISN Client <- Server
		\subitem Client -> ACK, ISN Server -> Server
	\end{itemize}
		
		
		
		
	
	
	
	
	
	
	
	
	
	
	
	
	
	
	
	
	
	
	
	
	
	
	
\end{document}