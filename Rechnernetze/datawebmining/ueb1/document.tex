\documentclass[11pt,a4paper,parskip=half ]{scrartcl}
\usepackage[utf8]{inputenc}
\usepackage[ngerman]{babel}
\usepackage{amsmath}
\usepackage{amsfonts}
\usepackage{amssymb}
\usepackage{graphicx}
\usepackage{xcolor}
\author{Aaron Winziers - 1176638; Michael Wolz - 1195270}
\title{Data and Webmining WS 2018/2019\\\LARGE{Übungsblatt 1}}

\begin{document}
	\maketitle
	
	\section*{Aufgabe 1}
	
	\subsection*{a)}
	\begin{enumerate}
		\item Verstehen der Anwendungsdomäne
		\subitem Identifikation der verfügbaren Daten
		\subitem Festlegung des KDD Ziels
		
		\item Zieldatenfestlegung (Selektion)
		\subitem Festlegung der Datenbanken, Datensätze, Attribute die untersucht werden sollen.
		
		\item Vorverarbeitung und Datenbereinigung
		\subitem Erkennung und Eliminierung von Datenfehlern (Ausreißern) und von fehlenden Einträgen
		
		\item Datenreduktion und Projektion (Transformation)
		\subitem Identifikation der nützlichen Attribute für die KDD Aufgabe
		\subitem Reduktion der Dimension (Attribute)
		\subitem Berechnung abgeleiteter Attribute
		\subitem Reduktion der zu bearbeitenden Daten (Sampling)
		
		\item Auswahl der Data Mining Aufgabenklasse
		\subitem um welche Art von Data Mining Aufgabe handelt es sich, z.B. Klassifikation, Regression, Assoziation, Clustering
		
		\item Wahl des Data Mining Algorithmus
		\subitem für den gewählt Aufgabenklasse: bestimme einen geeigneten Algorithmus
		\subitem je nach Algorithmus: Bestimmung von Modellparametern
		
		\item Data Mining durchführen
		\subitem Anwendung des Algorithmus auf den vorverarbeiteten Daten
		
		\item Interpretation
		\subitem gefundene Muster werden interpretiert
		\subitem ggf. weitere Iteration und Wiederholung der Schritte 1-7
		
		\item Konsolidierung des KDD Ergebnisses
		\subitem Präsentation der Ergebnisse und Dokumentation
	\end{enumerate}
	
	\subsection*{b)}
	
	
	\subsection*{c)}
	
	
	\section*{Aufgabe 2}
	\subsection*{a)}
	
	\textbf{Klassifikation}\\
	\begin{itemize}
		\item Einordnung einer Beobachtung in eine von n Klassen
		\item Klassenbeschreibung = gelernte Regeln
		\item Regeln werden angeendet um neue Beobachtungen zu klassifizieren
	\end{itemize}
	
	\textbf{Regression}\\
	\begin{itemize}
		\item Lernen einer Funktion zur Abbildung einer Beobachtung auf einen Zahlenwert
		\item Approximation einer Datenmenge durch mathematische Funktion
		\item Ziel: möglichst gute Datenerfassung, d.h. möglichst geringe Fehler machen
		\item Durch Funktion können Vorhersagen getroffen werden
	\end{itemize}
	
	\textbf{Clustering}\\
	\begin{itemize}
		\item Einteilung von Datensätzen in Gruppen, z.B. disjunkte Gruppen oder hierarchische Gruppierungen
		\item Ermittlung von Klassen für unklassifizierte Datenpunkte
		\item Objekte möglichst Homogen
		\item Cluster untereinander möglichst heterogen
	\end{itemize}
	
	\textbf{Abhängigkeitsanalyse}\\
	\begin{itemize}
		\item Erkennen von gesetzmäßigen Abhängigkeiten in Daten, z.B. in Form von Regeln
	\end{itemize}
	
	\subsection*{b)}
	
	\textbf{Synthetisches Lernen}\\
	\begin{itemize}
		\item Bildung induktiver Schlüsse
		\item Wahrheitsgehalt der Schlussfolgerung nicht gesichert
		\item Neues Wissen wird geraten
	\end{itemize}
	
	\textbf{Analytisches Lernen}\\
	\begin{itemize}
		\item Bildung deduktiver Schlüsse
		\item wahrheitserhaltend: Schlussfolgerungen sind nachweislich korrekt
		\item Man lernt hier eigentlich kein wirklich neues Domänenwissen
	\end{itemize}
	
	\textbf{Lernen durch Analogie}\\
	\begin{itemize}
		\item Bildung analoger Schlüsse
		\item Wahrheitsgehalt der Schlussfolgerung nicht gesichert
		\item Basieren auf Ähnlichkeit
	\end{itemize}
		
	\section*{Aufgabe 3}
		\subsection*{a)}
			\textbf{Klassifikator}
			\begin{itemize}
				\item Ein Klassifikator für eine Menge M ist eine Abbildung $f : M \rightarrow I$, wobei $I$ eine Menge ist, die Indexmenge genannt wird.
				\item Wenn $I = \{0, 1\}$, dann heißt
				\subitem $P=\{x \in M | f(x) = 1\}$ die Menge der \textit{positiven} Elemente und
				\subitem $N=\{x \in M | f(x) = 0\}$ die Menge der \textit{negativen} Elemente
			\end{itemize}
		
			\textbf{Klassifikationsbeschreibung}
			\begin{itemize}
				\item Unterscheide: Klassifikator und Klassifikatorbeschreibung
				\item Verschiedene Möglichkeiten der Beschreibung:
					\subitem Aufzählung aller Elemente (nur bei endlicher Grundmenge)
					\subitem Angabe einer prädikatenlogischen Formel
					\subitem Angabe eines C-Programms
					\subitem ...
				\item Beachte
					\subitem Eine Klassifikatorbeschreibung bestimmt einen eindeutigen Klassifikator
					\subitem  Ein Klassifikator kann mehrere unterschiedliche Klassifikatorbeschreibungen besitzen 
			\end{itemize}
		
			\textbf{Konzeptbegriff}
			\begin{itemize}
				\item Konzepte sind Klassifikatorbeschreibungen
				\item Definition: Ein Konzept ist ein einstelliges Prädikat über einer Grundmenge M.
				\item Schreibweise:
					\subitem Für $x\in M$,  
					\subsubitem  K(x) : x gehört zum Konzept K (x ist ein positives Beispiel)
					\subsubitem $\neg$K(x) : x gehört nicht zum Konzept K (x ist ein negatives Beispiel)
			\end{itemize}
		
	
	\section*{Aufgabe 4}
	\subsection*{b)} Sie verändern sich nicht da die Precision und recall unabhängig von der Menge der True-Negatives(78) sind und die Menge der False-Positives (4) sich nicht verändert
	
	\subsection*{c)}
		\begin{itemize}
			\item[] \textbf{Recall = 1} : Kommt in diesem Fall vor wenn alle relevante Dokumente tatsächlich als relevant klassifiziert werden.
			\item[] \textbf{Precision = 1} : Kommt in diesem Fall vor wenn alle als relevant klassifizierte Dokumente tatsächlich relevant sind.
		\end{itemize}
	Wenn recall = 1, ist das Konzept vollständig da die vollständige Menge der relevanten Dokumente als relevant klassifiziert werden.
	
	
\end{document}