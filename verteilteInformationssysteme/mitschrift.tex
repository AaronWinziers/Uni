\documentclass[10pt,a4paper]{article}
\usepackage[utf8]{inputenc}
\usepackage{amsmath}
\usepackage{amsfonts}
\usepackage{amssymb}
\usepackage{graphicx}
\usepackage{hyperref}
\author{Mitschrift von Aaron Winziers\\
	Wintersemester 2019/20}
\title{Verteilte Informationssysteme}
\date{}
\begin{document}
	\maketitle
	\section{Organisation}
	Prüfungstermine:
	\begin{itemize}
		\item 19.02.2020
		\item 01.04.2020
	\end{itemize}

%%%%%%%%%%%%%%%%%%%%%%%%%%%%%%%%%%%%%%%%%%%%%%%%%%
%					Section						 %
%%%%%%%%%%%%%%%%%%%%%%%%%%%%%%%%%%%%%%%%%%%%%%%%%%
	
\section{Introduction}
\subsection{Lifecycle of an information system}
	\begin{itemize}
		\item Small company(startup) - Bookkeeping etc still possible on paper
		\item Growth - Information and data becomes too big for paper - database is
		needed
		\item Continued growth - Need for more databases or even specialized systems,
		more locations need more databases
	\end{itemize}
\subsection{What to do?}
	\begin{itemize}
		\item Data warehouse
		\item Distributed architecture
		\item Replication - Which locations need which data?
		\item Cloud computing
	\end{itemize}
	
	There are 2 important use cases that should be considered:
	transactional load(OLTP)(write heavy) and analytical load(OLAP)(read heavy)
	
\subsection{SQL vs NoSQL}
\subsubsection{SQL:}
	\begin{itemize}
		\item (+) declarative queries
		\item (+) consistency
		\item (+) guarantees
		\item (+) data independence
		\item (+) normalization
		\item (-) do not scale well with increasing load
		\item (-) features not always needed/are not used
		\item (-) rigid data structures
	\end{itemize}
\subsubsection{NoSQL:}
	\begin{itemize}
		\item ( ) not relational
		\item (+) scale well, are distributes in nature (more nodes = more
		performance)
		\item ( ) often w/o query language but with simple API
		\item (-) weak consistency models - distributed copies may not be identical
		\item (+) offer high performance
	\end{itemize}
	
	Both systems are typically combined into hybrid systems
	
\subsection{Reasons for distributing data}
	\begin{itemize}
		\item Cost and scalability - mainframes are difficult to extend
		\item Replications leads to higher availability
		\item Integration of different software modiles - prevent collisions
		\item Integration of legacy systems - old system can continue to exists parallel to new system
		\item New kinds of applications - esp. e-commerce
		\item Market forces
	\end{itemize}
	
\subsection{Why distribution?} Distributed data better corresponds to modern enterprise structures

%%%%%%%%%%%%%%%%%%%%%%%%%%%%%%%%%%%%%%%%%%%%%%%%%%
%					Section						 %
%%%%%%%%%%%%%%%%%%%%%%%%%%%%%%%%%%%%%%%%%%%%%%%%%%

\section{Distributed query processing}
\subsection{Important aspects}
	\paragraph{More replications} faster queries but slower updates
	\paragraph{Fragmentation} storing local data locally
	\paragraph{Parallelism} multiple queries can be performed at once, or a single query can be split into parts and executed at the same time
	\paragraph{Transparency} fragmentation, replication etc should not be need to be taken into account by the user
	
\subsection{Systems differ in terms of:}
	\begin{itemize}
		\item Degree of coupling
		\item Interconnection structure
		\item Interdependence of components
		\item Synchronization of components
	\end{itemize}

\subsection{Forms of distributed systems}
	\begin{itemize}
		\item Peer-to-Peer and file sharing
		\item Cloud computing
		\item Web services and the deep Web
		\item Semantic Web
		\item Big Data Analytics
	\end{itemize}

\subsection{Fallacies of distributed computing:}
	\begin{itemize}
		\item The network is reliable
		\item Latency is zero
		\item Bandwidth is infinite
		\item The network is secure
		\item Topology doesn't change
		\item There is one administrator
		\item Transport cost is zero
		\item The network is homogenous
		\item Location is irrelevant
	\end{itemize}

%%%%%%%%%%%%%%%%%%%%%%%%%%%%%%%%%%%%%%%%%%%%%%%%%%
%					Section						 %
%%%%%%%%%%%%%%%%%%%%%%%%%%%%%%%%%%%%%%%%%%%%%%%%%%
\section{Fragmentation and allocation in distributed database management systems}

%%%%%%%%%%%%%%%%%%%%%%%%%%%%%%%%%%%%%%%%%%%%%%%%%%
%					Section						 %
%%%%%%%%%%%%%%%%%%%%%%%%%%%%%%%%%%%%%%%%%%%%%%%%%%
\section{Replication and synchronization}

%%%%%%%%%%%%%%%%%%%%%%%%%%%%%%%%%%%%%%%%%%%%%%%%%%
%					Section						 %
%%%%%%%%%%%%%%%%%%%%%%%%%%%%%%%%%%%%%%%%%%%%%%%%%%
\section{Grid and cloud computing}

%%%%%%%%%%%%%%%%%%%%%%%%%%%%%%%%%%%%%%%%%%%%%%%%%%
%					Section						 %
%%%%%%%%%%%%%%%%%%%%%%%%%%%%%%%%%%%%%%%%%%%%%%%%%%
\section{Distributed transactions}

%%%%%%%%%%%%%%%%%%%%%%%%%%%%%%%%%%%%%%%%%%%%%%%%%%
%					Section						 %
%%%%%%%%%%%%%%%%%%%%%%%%%%%%%%%%%%%%%%%%%%%%%%%%%%
\section{Information integration}

%%%%%%%%%%%%%%%%%%%%%%%%%%%%%%%%%%%%%%%%%%%%%%%%%%
%					Section						 %
%%%%%%%%%%%%%%%%%%%%%%%%%%%%%%%%%%%%%%%%%%%%%%%%%%
\section{Distributed information retrieval}

%%%%%%%%%%%%%%%%%%%%%%%%%%%%%%%%%%%%%%%%%%%%%%%%%%
%					Section						 %
%%%%%%%%%%%%%%%%%%%%%%%%%%%%%%%%%%%%%%%%%%%%%%%%%%
\section{Parallel database systems}
	
\end{document}
