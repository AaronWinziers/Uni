\documentclass[10pt,a4paper]{article}
\usepackage[utf8]{inputenc}
\usepackage{amsmath}
\usepackage{amsfonts}
\usepackage{amssymb}
\usepackage{graphicx}
\usepackage{hyperref}
\author{Mitschrift von Aaron Winziers\\
	Wintersemester 2019/20}
\title{Verteilte Informationssysteme}
\date{}
\begin{document}
	\maketitle
	\section{Organisation}
	Prüfungstermine:
	\begin{itemize}
		\item 19.02.2020
		\item 01.04.2020
	\end{itemize}

%%%%%%%%%%%%%%%%%%%%%%%%%%%%%%%%%%%%%%%%%%%%%%%%%%
%					Section						 %
%%%%%%%%%%%%%%%%%%%%%%%%%%%%%%%%%%%%%%%%%%%%%%%%%%
	
\section{Introduction}
\subsection{Lifecycle of an information system}
	\begin{itemize}
		\item Small company(startup) - Bookkeeping etc still possible on paper
		\item Growth - Information and data becomes too big for paper - database is
		needed
		\item Continued growth - Need for more databases or even specialized systems,
		more locations need more databases
	\end{itemize}
\subsection{What to do?}
	\begin{itemize}
		\item Data warehouse
		\item Distributed architecture
		\item Replication - Which locations need which data?
		\item Cloud computing
	\end{itemize}
	
	There are 2 important use cases that should be considered:
	transactional load(OLTP)(write heavy) and analytical load(OLAP)(read heavy)
	
\subsection{SQL vs NoSQL}
\subsubsection{SQL:}
	\begin{itemize}
		\item (+) declarative queries
		\item (+) consistency
		\item (+) guarantees
		\item (+) data independence
		\item (+) normalization
		\item (-) do not scale well with increasing load
		\item (-) features not always needed/are not used
		\item (-) rigid data structures
	\end{itemize}
\subsubsection{NoSQL:}
	\begin{itemize}
		\item ( ) not relational
		\item (+) scale well, are distributes in nature (more nodes = more
		performance)
		\item ( ) often w/o query language but with simple API
		\item (-) weak consistency models - distributed copies may not be identical
		\item (+) offer high performance
	\end{itemize}
	
	Both systems are typically combined into hybrid systems
	
\subsection{Reasons for distributing data}
	\begin{itemize}
		\item Cost and scalability - mainframes are difficult to extend
		\item Replications leads to higher availability
		\item Integration of different software modiles - prevent collisions
		\item Integration of legacy systems - old system can continue to exists parallel to new system
		\item New kinds of applications - esp. e-commerce
		\item Market forces
	\end{itemize}
	
\subsection{Why distribution?} Distributed data better corresponds to modern enterprise structures

%%%%%%%%%%%%%%%%%%%%%%%%%%%%%%%%%%%%%%%%%%%%%%%%%%
%					Section						 %
%%%%%%%%%%%%%%%%%%%%%%%%%%%%%%%%%%%%%%%%%%%%%%%%%%

\section{Distributed query processing}
\subsection{Important aspects}
	\paragraph{More replications} faster queries but slower updates
	\paragraph{Fragmentation} storing local data locally
	\paragraph{Parallelism} multiple queries can be performed at once, or a single query can be split into parts and executed at the same time
	\paragraph{Transparency} fragmentation, replication etc should not be need to be taken into account by the user
	
\subsection{Systems differ in terms of:}
	\begin{itemize}
		\item Degree of coupling
		\item Interconnection structure
		\item Interdependence of components
		\item Synchronization of components
	\end{itemize}

\subsection{Forms of distributed systems}
	\begin{itemize}
		\item Peer-to-Peer and file sharing
		\item Cloud computing
		\item Web services and the deep Web
		\item Semantic Web
		\item Big Data Analytics
	\end{itemize}

\subsection{Fallacies of distributed computing:}
	\begin{itemize}
		\item The network is reliable
		\item Latency is zero
		\item Bandwidth is infinite
		\item The network is secure
		\item Topology doesn't change
		\item There is one administrator
		\item Transport cost is zero
		\item The network is homogeneous
		\item Location is irrelevant
	\end{itemize}

\subsection{Promises of distributed database systems}
	\subsubsection{Transparent data management} - Systems hide implementation details
	\paragraph{Data independence}
		\begin{itemize}
			\item Immunity of applications to changes in the definition and organization of
			data, and vice versa
			\item Logical data independence (changes in the schema definition)
			Application should still be running if additional attributes are added to a
			relation
			\item Physical data independence (changes to physical data organization)
			Hiding the physical data organization (relations, indexes)
		\end{itemize}
	\paragraph{Network Transparency}
	\paragraph{Replication Transparency}
	\paragraph{Fragmentation Transparency}
	
	\paragraph{Who should provide transparency?}	~\\
	Application - are implemented in a distributed fashion	- communicate with standard protocols	\\
	Operating system - provides network transparency on file system or protocol level	\\
	Database system - Transparent access to data at remote database instances - Requires splitting queries, transaction control, replication
	
	\subsubsection{Reliability}
	\begin{itemize}
		\item Compensating node failures through data copies
		\item Distributed transactions guarantee that:
		\begin{itemize}
			\item A sequence of operations is performed as an atomic action
			\item A consistent database state transforms into another consistent database state, even with multiple transactions occuring concurrently
		\end{itemize}
		\item Increased effort for updates, system may crash with server failure
	\end{itemize}
	
	\subsubsection{Improved performance}
	\begin{itemize}
		\item Fragmenting data in a way that enables data to be stored in close proximity to its points of use
		\item Distributed systems inherently have parallelism
		\begin{itemize}
			\item Inter-query - execution of multiple queries
			\item Intra-query - Parallel execution of sub-queries
		\end{itemize}
		\item Read-only vs Update access
		\begin{itemize}
			\item Query database (ad-hoc querying) and production database (for updates by application programs) - production database is copied into the query database in regular intervals
			\item Read-only access during regular hours, updates are batched and performed during off-hours
		\end{itemize}
	\end{itemize}
	
	\subsubsection{Easier system expansion}
	\begin{itemize}
		\item Ability to expand database size and/or decrease query time is a necessity
		\item Done by adding additional storage and processing power to the network
		\item A system of smaller computers is often cheaper than a larger single computer with equivalent power
	\end{itemize}

\subsection{Challenges}
	\begin{itemize}
		\item Distributed database design - fragmentation, replication, distribution
		\item Distributed query processing - maximizing cost-effectiveness of executing queries
		\item Distributed concurrency - Synchronizing access for integrity
		\item Reliability - Ensure consistency, detect failures, recover from failures
		\item Heterogeneous DBS - Translation between DBS - data model and data language
	\end{itemize}

\subsection{Standard architectures}
	\begin{itemize}
		\item Distributed information system - Applications communicate for data exchange
		\item DBMS - OS hides distribution
		\item Distributed DBS - Distribution handled by DBS
		\item Parallel DBS
		\begin{itemize}
			\item Data processing by simultaneous computers(multi-processor, special hardware)
			\item Increase performance by using multiple processing units
		\end{itemize}

	\end{itemize}

\subsection{Relationale Algebra}
	\begin{itemize}
		\item Union eliminates duplicates - SQL doesn't
	\end{itemize}

%%%%%%%%%%%%%%%%%%%%%%%%%%%%%%%%%%%%%%%%%%%%%%%%%%
%					Section						 %
%%%%%%%%%%%%%%%%%%%%%%%%%%%%%%%%%%%%%%%%%%%%%%%%%%
\section{Fragmentation and allocation in distributed database management systems}

%%%%%%%%%%%%%%%%%%%%%%%%%%%%%%%%%%%%%%%%%%%%%%%%%%
%					Section						 %
%%%%%%%%%%%%%%%%%%%%%%%%%%%%%%%%%%%%%%%%%%%%%%%%%%
\section{Replication and synchronization}

%%%%%%%%%%%%%%%%%%%%%%%%%%%%%%%%%%%%%%%%%%%%%%%%%%
%					Section						 %
%%%%%%%%%%%%%%%%%%%%%%%%%%%%%%%%%%%%%%%%%%%%%%%%%%
\section{Grid and cloud computing}

%%%%%%%%%%%%%%%%%%%%%%%%%%%%%%%%%%%%%%%%%%%%%%%%%%
%					Section						 %
%%%%%%%%%%%%%%%%%%%%%%%%%%%%%%%%%%%%%%%%%%%%%%%%%%
\section{Distributed transactions}

%%%%%%%%%%%%%%%%%%%%%%%%%%%%%%%%%%%%%%%%%%%%%%%%%%
%					Section						 %
%%%%%%%%%%%%%%%%%%%%%%%%%%%%%%%%%%%%%%%%%%%%%%%%%%
\section{Information integration}

%%%%%%%%%%%%%%%%%%%%%%%%%%%%%%%%%%%%%%%%%%%%%%%%%%
%					Section						 %
%%%%%%%%%%%%%%%%%%%%%%%%%%%%%%%%%%%%%%%%%%%%%%%%%%
\section{Distributed information retrieval}

%%%%%%%%%%%%%%%%%%%%%%%%%%%%%%%%%%%%%%%%%%%%%%%%%%
%					Section						 %
%%%%%%%%%%%%%%%%%%%%%%%%%%%%%%%%%%%%%%%%%%%%%%%%%%
\section{Parallel database systems}
	
\end{document}
