\documentclass[11pt,a4paper,parskip=half ]{scrartcl}
\usepackage[utf8]{inputenc}
\usepackage[ngerman]{babel}
\usepackage{amsmath}
\usepackage{amsfonts}
\usepackage{amssymb}
\usepackage{blkarray}
\usepackage{graphicx}
\usepackage{xcolor}
\usepackage{soul}

\usepackage{listings}
\lstset{
	numbers=left,
	showspaces=false,
	breaklines=true,
	tabsize=3,
	basicstyle=\ttfamily,
}

\author{Aaron Winziers - 1176638}
\title{Verteilte Informationssysteme WS 2019/20\\\LARGE{Übungsblatt 5}}

\begin{document}
	\maketitle
	
	\section*{Aufgabe 1}
	\section*{Aufgabe 2}
	\paragraph{(a)} Die Gesamte Optimierung zur Compile-Zeit berechnet zur Compile Zeit den vollständigen Ausführungsplan und nimmt an dass, Anwendungen immer die Gleichen Anfragemuster verwenden. Hierbei ist ein Vorteil dass, Anfragen sofort ausgeführt werden können. Ein NAchteil ist dass, die Modellierung sehr komplex ist.
	\paragraph{(b)} Die Vollständig dynamische Optimierung optimiert jede Anfrage individuell zur Laufzeit. Ein Vorteil davon ist dass, der aktuelle Zustand des Netzes berücksichtigt wird. Ein Nachteil dass, die Qualität der resultierende Pläne unvorhersagbar ist.
	\paragraph{(c)} Semi-dynamische Optimierung
	\paragraph{(d)} Hierarchische Optimierung mit globalem und lokalem Plan
	\paragraph{(e)} Zweischrittige hierarchische Optimierung
	
\end{document}
















