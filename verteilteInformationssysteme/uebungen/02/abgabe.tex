\documentclass[11pt,a4paper,parskip=half ]{scrartcl}
\usepackage[utf8]{inputenc}
\usepackage[ngerman]{babel}
\usepackage{amsmath}
\usepackage{amsfonts}
\usepackage{amssymb}
\usepackage{blkarray}
\usepackage{graphicx}
\usepackage{xcolor}

\usepackage{listings}
\lstset{
	numbers=left,
	showspaces=false,
	breaklines=true,
	tabsize=3,
	basicstyle=\ttfamily,
}

\author{Aaron Winziers - 1176638}
\title{Verteilte Informationssysteme WS 2019/20\\\LARGE{Übungsblatt 2}}

\begin{document}
	\maketitle
\section*{Aufgabe 2}
	\paragraph{a)} Die Vollständig- und Disjunktheitsregeln werden in diesem Fall beide offensichtlich erfüllt. Die Rekonstruierbarkeit folgt in dem Fall aus den anderen beiden Regeln.
	
	\paragraph{b)} Zwei Fragmente werden hier benötigt: 
	\begin{gather*}
		vorlesungen_{\leq5037} := \sigma_{gelesenvon\leq5037}(vorlesungen)	\\
		vorlesungen_{>5037} := \sigma_{gelesenvon>5037}(vorlesungen)
	\end{gather*}
	
	\paragraph{c)}

\section*{Aufgabe 3}
	\paragraph{a)}
	$ use =
	\begin{blockarray}{cccccc}
		 & vorname & name & fachgebiet & boss &  \\
		\begin{block}{c(ccccc)}
			q_{1} & 1 & 1 & 0 & 0 \\
			q_{2} & 1 & 1 & 0 & 1 \\
			q_{3} & 0 & 0 & 1 & 0 \\
			q_{4} & 1 & 0 & 1 & 1 \\
			q_{5} & 0 & 1 & 0 & 1 \\
		\end{block}
	\end{blockarray}
	$
	\paragraph{b)}$ qstat = (1,3,2,1,3)$
	
	\paragraph{c)}$ aff = 
	\begin{blockarray}{cccccc}
	& vorname & name & fachgebiet & boss &  \\
	\begin{block}{c(ccccc)}
	vorname 	& 5 & 4 & 1 & 4 \\
	name 		& 4 & 7 & 0 & 6 \\
	fachgebiet 	& 1 & 0 & 3 & 1 \\
	boss 		& 4 & 6 & 1 & 7 \\
	\end{block}
	\end{blockarray}
	$
	
\end{document}
