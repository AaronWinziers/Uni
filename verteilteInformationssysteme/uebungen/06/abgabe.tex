\documentclass[11pt,a4paper,parskip=half ]{scrartcl}
\usepackage[utf8]{inputenc}
\usepackage[ngerman]{babel}
\usepackage{amsmath}
\usepackage{amsfonts}
\usepackage{amssymb}
\usepackage{blkarray}
\usepackage{graphicx}
\usepackage{xcolor}
\usepackage{soul}
\usepackage{multirow}
\usepackage{lscape}

\usepackage{listings}
\lstset{
	numbers=left,
	showspaces=false,
	breaklines=true,
	tabsize=3,
	basicstyle=\ttfamily,
}

\author{Aaron Winziers - 1176638}
\title{Verteilte Informationssysteme WS 2019/20\\\LARGE{Übungsblatt 6}}

\begin{document}
	\maketitle
\section*{Aufgabe 1}
	Um die Kommunikationskosten minimal zu halten sollte folgende Reihenfolge verwendet werden: $((B\bowtie D)\bowtie A)\bowtie C$. Die Rechnungen um auf das Ergebnis zu kommen befinden sich in den folgenden Tabellen:
	\begin{landscape}
		\begin{center}
			\begin{table}[h]
	\begin{tabular}{|l|l|p{.3\textwidth}|p{.3\textwidth}|p{.3\textwidth}|}
		\hline
		\multirow{2}{*}{Join-Baum}         & \multirow{2}{*}{\begin{tabular}[c]{@{}l@{}}Größe des\\ Zwischenergebnisses\end{tabular}} & \multicolumn{3}{l|}{Minimale Kommunikationskosten mit Ergebnis in}                                                                                                                         \\ \cline{3-5} 
		&                                                                                          & $N_{1}$                                                          & $N_{2}$                                               & $N_{3}$                                                         \\ \hline
		$A\bowtie B$                       & $8$  & $16 (B \rightarrow N_{1})                                                      $&$ 24 (B \rightarrow N_{1}, A \bowtie)B \rightarrow N_{2})                                $&$ 32 (B \rightarrow N_{1}, A \bowtie)B \rightarrow N_{3})                                          $\\ \hline
		$A\bowtie C$                       & $32$ & $8 (C \rightarrow N_{1})                                                       $&$ 32 (A \rightarrow N_{2})                                           $&$ 64 (A \rightarrow N_{2}, A \bowtie)C\rightarrow N_{3})                                           $\\ \hline
		$B\bowtie C$                       & $64$ & $24 (B \rightarrow N_{1},C \rightarrow N_{1})                                               $&$ 0                                                     $&$ 24 (B \rightarrow N_{3},C \rightarrow N_{3})                                              $\\ \hline
		$B\bowtie D$                       & $8$  & $12 (D \rightarrow N_{2}, B \bowtie)D \rightarrow N_{1})                                          $&$ 4 (D \rightarrow N_{2})                                           $&$ 12 (D \rightarrow N_{2}, B \bowtie)D \rightarrow N_{3})                                          $\\ \hline
		

		\hline \hline
		$(A\bowtie B)\bowtie C$            & $4$  & $24 (B \rightarrow N_{1}, C \rightarrow N_{1})                                             $&$ 24 (B \rightarrow N_{1}, A \bowtie)B \rightarrow N_{2})                                $&$ 28 (B \rightarrow N_{1}, A \bowtie)B \rightarrow N_{2}, (A \bowtie)B) \bowtie)C \rightarrow N_{3})                       $\\ \hline
		$(A\bowtie B)\bowtie D$            & $4$  & $24 (B \rightarrow N_{1}, D \rightarrow N_{1})                                             $&$ 28 (B \rightarrow N_{1}, A \bowtie)B \rightarrow N_{2}, D \rightarrow N_{2})                        $&$ 32 (B \rightarrow N_{1}, A \bowtie)B \rightarrow N_{3})                                         $\\ \hline
		$(A\bowtie C)\bowtie B$            & $4$  & $24 (C \rightarrow N_{1}, B \rightarrow N_{1})                                              $&$ 28 (C \rightarrow N_{1}, B \rightarrow N_{1}, (A \bowtie)C) \bowtie)B \rightarrow N_{2})                $&$ 32 (C \rightarrow N_{1}, B \rightarrow N_{1}, (A \bowtie)C) \bowtie)B \rightarrow N_{3})                           $\\ \hline
		$(B\bowtie C)\bowtie A$            & $4$  & $36 (A \rightarrow N_{2}, (B \bowtie)C) \bowtie)A \rightarrow N_{1})                                    $&$ 32 (A \rightarrow N_{2})                                           $&$ 36 (A \rightarrow N_{2}, (B \bowtie)C)\bowtie)A \rightarrow N_{3})                                    $\\ \hline
		$(B\bowtie C)\bowtie D$            & $32$ & $32 (B \rightarrow N_{1},C \rightarrow N_{1}, D \rightarrow N_{1})                                      $&$ 4 (D \rightarrow N_{2})                                           $&$ 24 (B \rightarrow N_{3},C \rightarrow N_{3})                                              $\\ \hline
		$(B\bowtie D)\bowtie A$            & $4$  & $12 (D \rightarrow N_{2}, B \bowtie)D \rightarrow N_{1})                                          $&$ 16 (D \rightarrow N_{2}, B \bowtie)D \rightarrow N_{1}, (B \bowtie)D) \bowtie)A \rightarrow N_{2})              $&$ 20 (D \rightarrow N_{2}, B \bowtie)D \rightarrow N_{1}, (B \bowtie)D) \bowtie)A \rightarrow N_{3})                      $\\ \hline
		$(B\bowtie D)\bowtie C$            & $32$ & $20 (D \rightarrow N_{2}, B \bowtie)D \rightarrow N_{1}, C \rightarrow N_{1})                                  $&$ 4 (D \rightarrow N_{2})                                           $&$ 20 (D \rightarrow N_{2}, B \bowtie)D \rightarrow N_{3}, C \rightarrow N_{3})   $\\ \hline
	\end{tabular}
\end{table}
\begin{table}[h]
	\begin{tabular}{|l|l|p{.3\textwidth}|p{.3\textwidth}|p{.3\textwidth}|}
        \hline
        \multirow{2}{*}{Join-Baum}         & \multirow{2}{*}{\begin{tabular}[c]{@{}l@{}}Größe des\\ Zwischenergebnisses\end{tabular}} & \multicolumn{3}{l|}{Minimale Kommunikationskosten mit Ergebnis in}\\ \cline{3-5}
        &                                                                                          & $N_{1}$                                                          & $N_{2}$                                               & $N_{3}$                                                         \\ \hline
		$((A\bowtie B)\bowtie C)\bowtie D$ & $2$  & $30 (B \rightarrow N_{1}, A \bowtie)B \rightarrow N_{2}, D \rightarrow N_{2}, ((A \bowtie)B) \bowtie)C) \bowtie)D \rightarrow N_{1})          $&$ 28 (B \rightarrow N_{1}, A \bowtie)B \rightarrow N_{2}, D \rightarrow N_{2})                       $&$ 28 (B \rightarrow N_{1}, A \bowtie)B \rightarrow N_{2}, (A \bowtie)B) \bowtie)C \rightarrow N_{3})                       $\\ \hline
		$((A\bowtie B)\bowtie D)\bowtie C$ & $2$  & $30 (B \rightarrow N_{1}, A \bowtie)B \rightarrow N_{2}, D \rightarrow N_{2}, ((A\bowtie)B) \bowtie)D) \bowtie)C \rightarrow N_{1})           $&$ 28 (B \rightarrow N_{1}, A \bowtie)B \rightarrow N_{2}, D \rightarrow N_{2})                        $&$ 30 (B \rightarrow N_{1}, A \bowtie)B \rightarrow N_{2}, D \rightarrow N_{2}, ((A \bowtie)B)\bowtie)D) \bowtie)C \rightarrow N_{3})           $\\ \hline
		$((A\bowtie C)\bowtie B)\bowtie D$ & $2$  & $32 (C \rightarrow N_{1}, B \rightarrow N_{1}, D \rightarrow N_{1})                                     $&$ 32 (C \rightarrow N_{1}, B \rightarrow N_{1}, (A\bowtie)C) \bowtie)B \rightarrow N_{2}, D \rightarrow N_{2})          $&$ 32 (C \rightarrow N_{1}, B \rightarrow N_{1}, (A\bowtie)C) \bowtie)B \rightarrow N_{3})                            $\\ \hline
		$((B\bowtie C)\bowtie A)\bowtie D$ & $2$  & $38 (A \rightarrow N_{2}, D \rightarrow N_{2}, ((B \bowtie)C) \bowtie)A) \bowtie)D\rightarrow N_{1})                       $&$ 36 (A \rightarrow N_{2}, D \rightarrow N_{2})                                   $&$ 36 (A \rightarrow N_{2}, (B \bowtie)C)\bowtie)A \rightarrow N_{3})                                    $\\ \hline
		$((B\bowtie C)\bowtie D)\bowtie A$ & $2$  & $32 (B \rightarrow N_{1},C \rightarrow N_{1}, D \rightarrow N_{1})                                       $&$ 36 (D \rightarrow N_{2}, A \rightarrow N_{2})                                   $&$ 36 (B \rightarrow N_{1},C \rightarrow N_{1}, D\rightarrow N_{1}, ((B \bowtie)C) \bowtie)D)\bowtie)A \rightarrow N_{3})                $\\ \hline
		$((B\bowtie D)\bowtie A)\bowtie C$ & $2$  & $18 (D \rightarrow N_{2}, B \bowtie)D \rightarrow N_{1}, (B \bowtie)D) \bowtie)A\rightarrow N_{2}, ((B \bowtie)D) \bowtie)A) \bowtie)C \rightarrow N_{1}) $&$ 16 (D \rightarrow N_{2}, B \bowtie)D \rightarrow N_{1}, (B \bowtie)D) \bowtie)A \rightarrow N_{2})              $&$ 18 (D \rightarrow N_{2}, B \bowtie)D \rightarrow N_{1}, (B \bowtie)D) \bowtie)A \rightarrow N_{2},((B \bowtie)D) \bowtie)A) \bowtie)C \rightarrow N_{3}) $\\ \hline
		$((B\bowtie D)\bowtie C)\bowtie A$ & $2$  & $20 (D \rightarrow N_{2}, B \bowtie)D \rightarrow N_{1}, C \rightarrow N_{1})                                  $&$ 22 (D \rightarrow N_{2}, B \bowtie)D \rightarrow N_{1}, C \rightarrow N_{1}, ((B \bowtie)D)\bowtie)C) \bowtie)A \rightarrow N_{2}) $&$ 24 (D \rightarrow N_{2}, B \bowtie)D \rightarrow N_{1}, C \rightarrow N_{1}, ((B \bowtie)D)\bowtie)C) \bowtie)A \rightarrow N_{3})           $\\ \hline
	\end{tabular}
\end{table}
		\end{center}
	\end{landscape}


\section*{Aufgabe 2}
\paragraph{Ship Whole}~

	Auf beiden Rechnern werden 2 Nachrichten benötigt (eine um Daten zu verlangen, eine um Daten zu schicken).

	Um auf Rechner $N_{S}$ den Join zu durchführen müssen $card(R)*5(Attribute)=5.000$ Attributwerte versandt werden.

	Um auf Rechner $N_{R}$ den Join zu durchführen müssen $card(S)*5(Attribute)=50.000$ Attributwerte versandt werden.

\paragraph{Fetch as Needed}~
	
	Um auf Rechner $N_{S}$ den Join zu durchführen müssen $card(S)*2 (\text{jeweils Anfrage und Antwort})=2.000$ Nachrichten versendet werden und $card(S)\text{ein Wert pro Anfrage}+card(R\bowtie S)*5 = 6.000$ Attributwerte versandt werden.
	
	Um auf Rechner $N_{R}$ den Join zu durchführen müssen $card(R)*2 (\text{jeweils Anfrage und Antwort})=2.000$ Nachrichten versendet werden und $card(R)\text{ein Wert pro Anfrage}+card(R\bowtie S)*5 = 15.000$ Attributwerte versandt werden.
		
\section*{Aufgabe 3}

	Die Attribute werden wie Folgt zugeordnet:
	\begin{itemize}
		\item Projects (5 Attribute) : PNo, PName, Location, Duration, Budget
		\item Assignments (2 Attribute) : PNo, ENo
	\end{itemize}

\paragraph{(a)}2 Nachrichten, 1.500 Tupel * 2 Attribute = 3.000 Attributwerte
\paragraph{(b)}2 Nachrichten, 500 Tupel * 5 Attribute = 2.500 Attributwerte
\paragraph{(c)}2 Nachrichten, $card(Projects)*40\%*1\text{ Attribut} + (card(Projects)*40\%)*90\%*2\text{ Attribute} = 200 + 360 = 560$ Attributwerte
\paragraph{(d)}2 Nachrichten, $card(Assignments)*1\text{ Attribut} + card(Assignments)*40\%*5\text{ Attribute} = 1.500 + 3.000 = 4.500$ Attributwerte
\paragraph{(e)}
	
\end{document}
















