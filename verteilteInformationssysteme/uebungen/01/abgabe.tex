\documentclass[11pt,a4paper,parskip=half ]{scrartcl}
\usepackage[utf8]{inputenc}
\usepackage[ngerman]{babel}
\usepackage{amsmath}
\usepackage{amsfonts}
\usepackage{amssymb}
\usepackage{graphicx}
\usepackage{xcolor}

\usepackage{listings}
\lstset{
	numbers=left,
	showspaces=false,
	breaklines=true,
	tabsize=3,
	basicstyle=\ttfamily,
}

\author{Aaron Winziers - 1176638}
\title{Verteilte Informationssysteme WS 2019/20\\\LARGE{Übungsblatt 1}}

\begin{document}
	\maketitle
	
	\section*{Aufgabe 1}
	\paragraph{a)}
	\begin{enumerate}
		\item Kosten und Skalierbarkeit
		\item Replikation zur Verbesserung der Verfügbarkeit
		\item Integration verschiedener Softwaremodule
		\item Integration von Legacy Systemen
		\item Neue Anwendungen
		\item Zwänge des Marktes
	\end{enumerate}
	
	
	\paragraph{b)}
	Homogene - 
	\begin{itemize}
		\item Verwenden die gleiche Hardware
		\item Verwenden gleiches DBMS
		\item Datenstrukturen sind gleich
		\item Leicht erweiterbar durch weitere Instanzen
		\item Inflexibel
	\end{itemize}
	
	Heterogene -
	\begin{itemize}
		\item Kann aus verschiedene DBMS bestehen
		\item Erlaubt das integreieren von verschiedene bereits existierende DBMS
		\item Erlaubt späteres separieren von integrierte DBMS
		\item Kann auf verschiedener Hardware laufen
	\end{itemize}
	
	
	\paragraph{c)}
	\begin{itemize}
		\item Keine spezialisierte Server
		\item Alle Peers speichern Daten und erlauben Zugriff darauf
		\item Es gibt begrenzte Informationen über das Netz selber
		\begin{itemize}
			\item  Peers kenne nur ihre direkte Nachbarn
			\item  Es gibt kein globales Wissen
			\item  keine Zentrale Koordination
		\end{itemize}
	\end{itemize}
	
	
	\paragraph{d)} Die Begriffe erläutern welche Resourcen von verteilten Datenbanksystemen geteilt werden
	Shared Everything - Mehrere Rechner teilen sich Hauptspeicher und Sekundärspeicher
	\begin{itemize}
		\item Pro - Niedriger Komplexitätsgrad
		\item Kontra - Skalierbarkeit schwieriger zu realisieren
	\end{itemize}
	Shared Disk - Mehrere Rechner - jeder mit eigenem Speicher - teilen sich gemeinsamen Sekundärspeicher
	\begin{itemize}
		\item Pro - Einheitliche Daten
		\item Kontra - Zugriffe auf die Disk müssen koordiniert werden
	\end{itemize}
	Shared Nothing - Weder Hauptspeicher noch Sekundärspeicher werden zwischen den Rechnern geteilt
	\begin{itemize}
		\item Pro - Es gibt kein Single Point of Failure
		\item Kontra - Daten müssen aktiv geprüft und korrigiert werden um einheitlich zu bleiben
	\end{itemize}
	
	\section*{Aufgabe 2}
	
	\paragraph*{a)} $\pi_{name, fachgebiet}(\sigma_{vorname = \text{'Philosoph'}}(assistenten))$
	
	\paragraph{b)} $\pi_{vorname, name}(\sigma_{(s.semester \geq 3) \wedge (p.note = 1.0)}(studenten\bowtie pruefen))$ (Natürlicher Join, wegen s.matrnr und p.matrnr)
	
	\paragraph{c)}Ausgegeben werden die Vornamen und Namen aller Studenten die die Prüfung zur Vorlesung "Ethik" geschrieben haben
	
	\paragraph{d)}~
	\begin{lstlisting}[language=SQL]
SELECT professoren.vorname, 
		professoren.name, 
		vorlesungen.titel
FROM professoren
	JOIN vorlesungen 
		ON professoren.persnr = vorlesungen.gelesenvon
	JOIN voraussetzen 
		ON vorlesungen.vorlnr = voraussetzen.fortgeschrittennr
WHERE professoren.schwerpunkt = 'Philosophie' 
	AND vorlesungen.sws = 4
	\end{lstlisting}
	
	\paragraph{e)}$\pi_{titel}(\sigma_{vorlesungen.vorlnr = (\pi_{vorlnr}(vorlesungen) - (\pi_{vorlnr}(hoeren) \cup \pi_{vorlnr}(pruefen)))}(vorlesungen))$
	
	\paragraph{f)}
	\begin{lstlisting}[language=SQL]
SELECT pruefungen.matrnr,
		pruefungen.titel,
		leser.lesername.name, 
		pruefungen.pruefername
FROM 
	(SELECT persnr AS leserpersnr, 
			name AS lesername
	 FROM professoren) AS leser
	JOIN leser 
		ON 
		(SELECT professoren.persnr AS prueferpersnr, 
				professoren.name AS pruefername,
				pruefen.matrnr,
				vorlesungen.titel
		 FROM professoren
			JOIN pruefen 
				ON professoren.persnr = pruefen.persnr
			JOIN vorlesungen 
				ON vorlesungen.gelesenvon = professoren.persnr
				) AS pruefungen
WHERE leserpersnr <> prueferpersnr
	\end{lstlisting}
	
	\begin{lstlisting}[language=SQL]
	
	\end{lstlisting}
	
	
	
	
	
	
	\section*{Aufgabe 3}
	\paragraph{a)} Fragmentierung ist die Aufteilung einer Relation in kleinere Teile. Dies kann entweder Horizontal(Tupel werden aufgeteilt), Vertikal(Attribute werden aufgeteilt), oder Hybrid(eine Kombination aus Vertikal und Horizontal) erfolgen.
	
	\paragraph{b)} Allokation ist das Vertielen von Datenfragmenten auf verschiedene Knoten in einem DBS.
	
	\paragraph{c)} Wenn die Vollständigkeitsregel verletzt ist, gibt es ein oder mehrere Tupel die in keinem Fragment sind. Damit sind die Daten nicht mehr \textit{Vollständig}
	
	\paragraph{d)} Die Disjunktheitsregel ist deswegen wichtig weil mit steigender Anzahl an Tupel die auf mehrere Knoten verteilt sind, steigt auch die Komplexität der Updates auf diese Einträge.
	
	\paragraph{e)}
	\begin{enumerate}
		\item Redundanz eliminieren
		\item Anzahl der Zugriffe auf einzelne Knoten
		\item Menge der Daten pro Knoten
		\item Geografische Nähe zum Ort an dem Daten benötigt werden
		\item Reduktion der Daten die zurückgegeben werden bei Anfragen
	\end{enumerate}
	
	
	
\end{document}
















