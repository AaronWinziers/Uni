\documentclass[10pt,a4paper]{article}
\usepackage[latin1]{inputenc}
\usepackage{amsmath}
\usepackage{amsfonts}
\usepackage{amssymb}
\usepackage{graphicx}
\usepackage{hyperref}
\author{Mitschrift von Aaron Winziers\\
		Wintersemester 2019/20}
\title{Dateisysteme \& Implementierung von Datenbanksystemen}
\begin{document}
	
	
\maketitle
\section{Organisation}
\href{http://dblps.uni-trier.de/~ley/dbimpl19}{User: student, Passwort: Jaeckel}	\\
Pr�fungstermine:
\begin{itemize}
	\item 12.02.2020
	\item 06.04.2020
\end{itemize}
Zwei Vortr�ge:
\begin{itemize}
	\item Eigener Vortrag
	\item Mit Folien von \href{https://www.youtube.com/user/jensdit}{Prof. J. Dittrich}
\end{itemize}
	\subsection{Motivation}
		 \begin{itemize}
		 	\item �nderungen in Arten/Verf�gbarkeit von Speicher (SSD, In-Memory DB) f�hren zu �nderungen in Technologien
		 \end{itemize}
	
\section{Einleitung}
	\begin{itemize}
		\item Was ist eine Datei?
		\begin{itemize}
			\item Eine Reihe von Bytes
			\item ggf. mit Eigenshaften ((Zugriffs-)Rechte, Timestamp, etc.)
			\item Inhalt ist dem System egal
			\item Kennt keine Semantik
		\end{itemize}
		\item VS. DB-System
		\begin{itemize}
			\item Bietet Struktur
			\item Wei� �ber sein Inhalt
			\item Physische Ebene wird als Programmierer abstrahiert
		\end{itemize}
		\item Objekt-Orientierte DBS haben sich nicht durchgesetzt
		\begin{itemize}
			\item OO konnte man nicht mit Mengen-Orientierte Weise vereinbaren
			\item F�hrte aber zu variiertere Datentypen gef�hrt
		\end{itemize}
		\item Deduktive Systeme
		\begin{itemize}
			\item Von Japaner kreiert
			\item Auf Logik basiert
			\item F�hrte zu Iteration in Query-Builder
		\end{itemize}
	\end{itemize}

\section{Dateisysteme}
\section{NTFS, XFS, Journaling FS, ...}
\section{Pufferverwaltung}
\section{Satzverwaltung/Zugriffspfade}
\section{Transaktionen}
\section{Architektur}
\section{Query Processor}


\end{document}