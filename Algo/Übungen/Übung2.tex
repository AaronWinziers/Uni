\documentclass[10pt,a4paper]{article}
\usepackage[utf8]{inputenc}
\usepackage{amsmath}
\usepackage{amsfonts}
\usepackage{amssymb}
\usepackage{graphicx}
\usepackage{listings}

\begin{document}
	\begin{flushright}
		Aaron Winziers, 1176638	\\
		Guillaume Kaufhold, 846375
	\end{flushright}
	\begin{center}
		\underline{\textbf{Algorithmen und Datenstrukturen Übung 2}}
	\end{center}
	\section{Aufgabe 2.1}
	\begin{lstlisting}
BINSEARCH(int lower, int upper, int x){
	int mid = roundDown(upper+lower);
	if (upper < lower){
		return 0;
	} else if(A[mid] == x){
		while(A[mid-1]==x){
			mid-1}
		return mid;
	} else if (A[mid] > x){
		BINSEARCH(lower, mid-1, x);
	} else{
		BINSEARCH(mid+1, upper, x);
	}
}
	\end{lstlisting}
	\section{Aufgabe 2.2}
	\begin{itemize}
		\item[a)] Ja, da $\lim\limits_{x\rightarrow\infty}\frac{2^{x}}{4^{}}=0$
		\item[b)] Nein, da egal wie $c$ gesetzt wird, wird die Aussage $1\geq c*n$ immer ab $n>c^{-1}$ nicht mehr gelten.
		\item[c)] Ja, bei jedem $c>\frac{1}{37}$ gilt die Aussage $|\sin(x)|\leq c*37$.
		\item[d)] Ja, die Funktion $(n+3)^{2}$ ist stets schneller wachsend
	\end{itemize}

	\section{Aufgabe 2.3}
	\begin{lstlisting}
Merge(A,l,m,r){
	int B[]
	for (int i = l; i \leg r; i++){
		B[i] = A[i];	
	}
	int j=l;
	int k=m+l;
	for (int i = l; i \leg r; i++){
		if (k>r) {
			A[i] = B[j];
			j++;
		} else if (j\leq m && B[j]<B[k]){
			A[i] = B[j];
			j++
		}else if /j\leq m && B[j]\leqB[k]){
			A[i] = B[k];
			k++;
		} else {
			A[i] = B[k];
			k++;
		}
	}
}
	\end{lstlisting}
	
	
	
	
	
	
	
	
	
	
	
	
	
	
	
	
	
	
	
	
	
	
	
	
	
	
	
\end{document}