\documentclass[10pt,a4paper]{article}
\usepackage[utf8]{inputenc}
\usepackage{amsmath}
\usepackage{amsfonts}
\usepackage{amssymb}
\usepackage{graphicx}
\usepackage{listings}

\begin{document}
	\begin{flushright}
		Aaron Winziers, 1176638	\\
		Guillaume Kaufhold, 846375
	\end{flushright}
	\begin{center}
		\underline{\textbf{Algorithmen und Datenstrukturen Übung 5}}
	\end{center}
	\section*{Aufgabe 5.1}		
	
	\section*{Aufgabe 5.2}
	\paragraph{BubbleSort} Stabil - da die Elemente mit gleichen Schlüsseln sich nicht gegenseitig 'überspringen' wird die Originale Struktur erhalten.
	
	\paragraph{HeapSort} Instabil - durch das bauen des Heaps wird die originale Reihenfolge. 'zerstört'
	
	\paragraph{MergeSort} Stabil - durch das Auseinanderziehen des Arrays wird die Reihenfolge bis zu dem Punkt wo sortiert wird die Reihenfolge erhalten.
	
	
	\section*{Aufgabe 5.3}
		Zum Beispiel beim Sortieren eines Adressbuchs. Das Adressbuch muss natürlich nach Namen sortiert sein, jedoch ist es immer noch wichtig das unter den Namen auch die Vornamen sortiert sind. Wenn ein Algorithmus nicht stabil ist würde die Reihenfolge der Vornamen kaputt gehen nach dem Sortieren nach Namen.
		
\end{document}
