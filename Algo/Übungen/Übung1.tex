\documentclass[10pt,a4paper]{article}
\usepackage[utf8]{inputenc}
\usepackage{amsmath}
\usepackage{amsfonts}
\usepackage{amssymb}
\usepackage{graphicx}
\usepackage{listings}
\author{Aaron Winziers}
\begin{document}
	\begin{flushright}
		Aaron Winziers	\\
		1176638
	\end{flushright}
	\begin{center}
		\underline{\textbf{Algorithmen und Datenstrukturen Übung 1}}
	\end{center}
	\section{Aufgabe 1.1}
	\subsection{(a)}
	\begin{lstlisting}
int min = 1;
for (int i = 2; i < n; i++)
	if A[i] < A[min]
		min = i;
return min;		
	\end{lstlisting}
	
	\subsection{(b)} Der Algorithmus muss im schlimmsten Fall 4n+1 Rechenschritte durchführen.
	
	\subsection{(c)} Der Algorithmus muss immer die gleiche Anzahl an Rechenschritte durchführen, da selbst wenn die Zahl in Index 1 die Kleinste ist müssen alle andere Zahlen überprüft werden.
	
	\section{Aufgabe 1.2}
	\begin{lstlisting}
for (int i = 1; i < n; i++){
	min = FINDMINUMUM(A[i..n]);
	if (i != min){
		int temp = A[i];
		A[i] = A[min];
		A[min] = temp;
	}
}	
	\end{lstlisting}
	re
	Der Algorithmus benötigt ungefähr $n(5+n) = 5n+n^{2}$ Rechenschritte um im schlimmsten Fall das Array zu sortieren
	
	\section{Aufgabe 1.3}
	Übergeben werden die größte und kleinste Indizes
	\begin{lstlisting}
BINSEARCH(int lower, int upper, int x){
	int mid = roundDown(upper+lower);
	if (upper < lower){
		return 0;
	} else if(A[mid] == x){
		while(A[mid-1]==x){
			mid-1}
		return mid;
	} else if (A[mid] > x){
		BINSEARCH(lower, mid-1, x);
	} else{
		BINSEARCH(mid+1, upper, x);
	}
}
	\end{lstlisting}
	
\end{document}