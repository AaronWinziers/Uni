\documentclass[10pt,a4paper]{article}
\usepackage[utf8]{inputenc}
\usepackage{amsmath}
\usepackage{amsfonts}
\usepackage{amssymb}
\usepackage{graphicx}
\usepackage{tcolorbox}
\usepackage{lipsum}
\usepackage{algorithmic}
\usepackage{listings}
\usepackage{color}

\definecolor{mygreen}{rgb}{0,0.6,0}
\definecolor{mygray}{rgb}{0.5,0.5,0.5}
\definecolor{mymauve}{rgb}{0.58,0,0.82}

\lstset{ %
	backgroundcolor=\color{white},   % choose the background color; you must add \usepackage{color} or \usepackage{xcolor}; should come as last argument
	basicstyle=\footnotesize,        % the size of the fonts that are used for the code
	breakatwhitespace=false,         % sets if automatic breaks should only happen at whitespace
	breaklines=true,                 % sets automatic line breaking
	captionpos=b,                    % sets the caption-position to bottom
	commentstyle=\color{mygreen},    % comment style
	deletekeywords={...},            % if you want to delete keywords from the given language
	escapeinside={(*}{*)},          % if you want to add LaTeX within your code
	extendedchars=true,              % lets you use non-ASCII characters; for 8-bits encodings only, does not work with UTF-8
	frame=single,	                   % adds a frame around the code
	keepspaces=true,                 % keeps spaces in text, useful for keeping indentation of code (possibly needs columns=flexible)
	keywordstyle=\color{blue},       % keyword style
	language=Java,                 % the language of the code
	morekeywords={*,...},            % if you want to add more keywords to the set
	numbers=left,                    % where to put the line-numbers; possible values are (none, left, right)
	numbersep=5pt,                   % how far the line-numbers are from the code
	numberstyle=\tiny\color{mygray}, % the style that is used for the line-numbers
	rulecolor=\color{black},         % if not set, the frame-color may be changed on line-breaks within not-black text (e.g. comments (green here))
	showspaces=false,                % show spaces everywhere adding particular underscores; it overrides 'showstringspaces'
	showstringspaces=false,          % underline spaces within strings only
	showtabs=false,                  % show tabs within strings adding particular underscores
	stepnumber=2,                    % the step between two line-numbers. If it's 1, each line will be numbered
	stringstyle=\color{mymauve},     % string literal style
	tabsize=2,	                   % sets default tabsize to 2 spaces
	title=\lstname                   % show the filename of files included with \lstinputlisting; also try caption instead of title
}



\author{Aaron Winziers}
\title{Algo Lernen}
\begin{document}
	{\huge\textbf{Look up runtimes}}
	\begin{itemize}
		\item Sorts:
		\begin{itemize}
			\item sort in $O(n+k)$ time (Feld)(Counting sort)
			\item sort in $O(n)$ (List)
			\subitem sort alphabetical (bucketsort lexicograpical sort)
		\end{itemize}
		\item Graphen:
		\begin{itemize}
			\item reversen Graph (j,i to i,j)
			\item Komplement des Graph (Adjazenzmatrix 1 to 0, 0 to 1)
			\item 2014.5
			\item Baum-, Rückwärts-, Vorwärts-, oder Querkante
		\end{itemize}
	\end{itemize}

\section{Sorts}
	\subsection{Heapsort}
		\subsubsection{Aufbauphase}
			\begin{lstlisting}
for i=(*$\lfloor\frac{n}{2}\rfloor$*)downto 1
	Sink(i,n)
			\end{lstlisting}
		\subsubsection{Heapify}
			\begin{lstlisting}
Sink(i,r){
	x = A[i];
	j = 2*i;
	while (j(*$\leq$*) r){
		if (j+1 (*$\leq$*) r){
			if (A[j+1]>A[j]){
				j = j+1;
			}
		}
		if (x(*$\geq$*) A[j]){
			break;
		}
		A[i] = A[j];
		i = j;
		j = 2*i;
	}
	A[i] = x;
}			
			\end{lstlisting}
		\subsubsection{Selektionsphase}
			\begin{lstlisting}
r = n
while (r > 1){
	A[1] <=> A[r];
	r = r-1;
	Heapify(1,r)
}
			\end{lstlisting}
	\subsection{Quicksort}
		\begin{lstlisting}
Quicksort(l,r){
	if (L(*$\geq$*)r)
		return;
	x = A[l]		pivot
	i = l+1
	j = r
	repeat
		while i(*$\leq$*)r && A[i]<x
			i++0
		while j(*$\geq$*)l+1 && A[j](*$\geq$*)x
			j--
		if i<j
			A[i]<=>A[j]
	until i>j
	A[l]<=>A[j]
	Quicksort(l,j-1)
	Quicksort(j+1,r)
}
		\end{lstlisting}
			
\section{Bäume}
	\subsection{Aufbau}
		\begin{lstlisting}
BaueBaum(A,l,r){
	if (l>r)
		return 0;
	m = (*$\lfloor\frac{l+r}{2}\rfloor$*);
	p = new bintree_node();
	p.key = A[m];
	p.left = BaueBaum(l,m-1);
	p.right = BaueBaum(m+1,r);
	return p;
}
		\end{lstlisting}
	
	\subsection{Lookup}
		\begin{lstlisting}
Lookup(x)
	if (p = null)
		return null;
	while (p ist kein Blatt)
		if x
		\end{lstlisting}
	\subsection{Insert}
		\begin{lstlisting}
Insert(Node root, int key, int value){
	if (root = null) 
		root = new Node(key, value);
	else if (key < root.key)
		root.left = insert(root.left, key, value);
	else  // key >= root.key
		root.right = insert(root.right, key, value);
	return root;
}
		\end{lstlisting}
		
	\subsection{Delete}
		\begin{lstlisting}
Delete(x){
	v = lookup(x);
	(Sei p vater von v)
	Fall 1: (v ist Blatt){
		if (v = p.left)
			p.left = null;
		else
			p.right = null;
	}
	Fall 2: (v hat ein Kind w){
		if (v = p.left)
			p.left = w;
		else
			p.right = w;
	}
	Fall 3: (v hat 2 Kinder)
		u = v.left;
		while (u.right!=null)
			u = u.right;
		v.key = u.key;
}
		\end{lstlisting}
	
	\subsection{Print}
		\begin{lstlisting}
Print(node n){
	if node.left != null
		print(node.left)
	schreibe(node.key)
	if node.right != null
		print(node.right)
}
		\end{lstlisting}
	\subsection{Max/Min Key}
		\begin{lstlisting}
Max(){
	u = rootNode
	while u.right != null
		u = u.right
	return u;
}


Min(){
	u = rootNode
	while u.left != null
		u = u.left
	return u;
}
		\end{lstlisting}
	\subsection{Rotate}
		\subsubsection{Single} %https://stackoverflow.com/questions/4597650/code-with-explanation-for-binary-tree-rotation-left-or-right
			\begin{lstlisting} 
RotateRight(node u){
	w = u.parent
	v = u.right
	b = v.left
	v.parent = w
	if w = null
		v = root
	else if u = w.left 
		w.left = v
	else
		w.right = v
	u.parent = v
	v.left = u
	u.right = b
	if b!=null
		b.parent = u
}
			\end{lstlisting}
		\subsubsection{Double}
			\begin{lstlisting}
DoubleRR(node u){
	v = u.left
	rotateLeft(v)
	rotateRight(u)
}

DoubleRL(node u){
v = u.right
rotateRight(v)
rotateLeft(u)
}
			\end{lstlisting}
		\subsection{Depth-First Search}
			\begin{lstlisting}
DFS(G, v){
	v als beschriften
	for all adjacent nodes w to v do
		if node  is not labeled discovered 
			DFS(G,w)
}
			\end{lstlisting}
		\subsection{Breadth-First Search}
			\begin{lstlisting}
BFS(G,v){
	Q:= queue initialized with root
	while Q.notEmpty()
		current = Q.dequeue()
		if current = goal
			return current
		for each node n that is adjacent to current
			if n is not discovered
				label n as discovered
				n.parent = current
				Q.enqueue(n)
}
			\end{lstlisting}


\section{Graphen}


\section{Definitionen}
	\subsection{Topologische Sortierung}
		Eine Abbildung $ord:V\rightarrow\{1,...,n\}$	\\
		Es gilt zudem:
		\begin{itemize}
			\item $ord$ ist injektiv
			\item $\forall(v,w)\in E:ord(v)<ord(w)$
		\end{itemize}
		






\end{document}