\documentclass[11pt,parskip=full]{scrartcl}
\usepackage[utf8]{inputenc}
\usepackage{graphicx}
\usepackage{hyperref}
\usepackage{setspace}

\newcommand{\HRule}[1]{\rule{\linewidth}{#1}}
\usepackage{float}


%%%%%%%%%%%%%%%% Code listings %%%%%%%%%%%%%%%%
\usepackage{listings}
\usepackage[dvipsnames]{xcolor}

\definecolor{javared}{rgb}{0.6,0,0} % for strings
\definecolor{javagreen}{rgb}{0.25,0.5,0.35} % comments
\definecolor{javapurple}{rgb}{0.5,0,0.35} % keywords
\definecolor{javadocblue}{rgb}{0.25,0.35,0.75} % javadoc
\definecolor{editorbg}{HTML}{F3F3F3}

\lstset{language=Java,
	backgroundcolor=\color{editorbg},
	basicstyle=\footnotesize,
	keywordstyle=\color{javapurple}\bfseries,
	stringstyle=\color{javared},
	commentstyle=\color{javagreen},
	morecomment=[s][\color{javadocblue}]{/**}{*/},
	numbers=left,
	numberstyle=\small\color{black},
	stepnumber=1,
	numbersep=10pt,
	tabsize=4,
	showspaces=false,
	showstringspaces=false,
	frame=single,
	captionpos=b,
	xleftmargin=.05\textwidth,
	xrightmargin=.05\textwidth,
	aboveskip=20pt,
	breaklines=true,
	postbreak=\mbox{\textcolor{red}{$\hookrightarrow$}\space},
	escapechar=!
	}

\begin{document}

\begin{titlepage}
  \begin{center}
    \vspace*{2cm}
    \HRule{1pt} \\
    \vspace{.5 cm}
    \textbf{\Huge Server-Side Renderer}
    \HRule{2pt} \\ [1cm]

    \vspace{1.5cm}

    \Large{
      Ausarbeitung zur Lösung des Blocks Spiels mittels Server-seitigem Rendering
    }
    \vspace{1.5cm}

    \textbf{\Large Benedikt Lüken-Winkels} - \large Matrikelnumber \\
    [3pt]
    \textbf{\Large Aaron Winziers} - \large 1176638

    \vfill

    \includegraphics[width=0.7\textwidth]{Logo-Uni-Trier}\\
    [1cm]
    Lehrstuhl für Systemsoftware und Verteilte Systeme\\
    Universität Trier\\
    01.04.2019

  \end{center}
\end{titlepage}
\newpage

\section{Einleitung}

Im Rahmen der Vorlesung ``Spieleprogrammierung'' im Sommersemester 2019 sollten verschiedene Gebiete der Entwicklung von Computerspielen näher untersucht werden. Hierfür wurden verschiedene Gruppen gebildet, welche jeweils ein Teilgebiet oder Konzept untersuchen und in dem Spiel \textit{Blocks} implementiert werden.\textit{Blocks} war in diesem Fall ein simples ``world building'' Spiel in dem man sich durch eine Welt mit einem Avatar bewegen kann und in dieser Welt Blöcke setzen und löschen kann.

Diese Ausarbeitung behandelt die Implementierung eines Server-Side Renderers in der Spile-Engine \href{https://unity.com/}{Unity} und erläutert die Entwicklung derselben.

\section{Server-Side Renderer}

\section{Aussicht}
Es gibt einige Möglichkeiten die Implementierung auf verschiedene Weisen zu verbessern.

\subsection{Minimaler Client}
Aktuell ist der Client auch in Unity implementiert und verbraucht damit, vor allem bei dem Starten des Programms, mehr Ressourcen als notwendig wären. Da der Client nur wenig Funktionalität bieten muss, und somit viel weniger benötigt als das was Unity bieten kann, wäre es möglich und erstrebenswert ein noch weiter minimierten Client zu entwickeln.

\subsection{Das Netzwerkalgorithmus}

\subsection{EncodeToPNG}


\section{Anhang}

Der gesamte Programmcode ist auf GitHub verfügbar.

$[1]$ HtHub

\end{document}
