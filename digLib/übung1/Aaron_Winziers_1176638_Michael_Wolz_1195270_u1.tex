\documentclass[11pt,a4paper,parskip=half ]{scrartcl}
\usepackage[utf8]{inputenc}
\usepackage[ngerman]{babel}
\usepackage{amsmath}
\usepackage{amsfonts}
\usepackage{amssymb}
\usepackage{graphicx}
\usepackage{xcolor}
\author{Aaron Winziers - 1176638; Michael Wolz - 1195270}
\title{Digital Libraries WS 2018/2019\\\LARGE{Übungsblatt 1}}

\begin{document}
	\maketitle
	
	\section*{Aufgabe 1}
	
	\begin{itemize}
		\item \textbf{Suche einer bestimmten Publikation}
		\item[] Ja, die Suche nach bestimmten Publikationen ist sehr ausführlich möglich. In der erweiterten Suche wird die Möglichkeit gegeben sehr spezifisch nach bestimmten Kriterien zu suchen. 
		\item \textbf{Suche nach „passenden“ Publikationen auf Basis von}
		\begin{itemize}
			\item \textbf{Exakten oder inexakten Metadaten}
			\item[] Ja, sowohl exakte als auch inexakte Metadaten
			\item \textbf{Inhaltlichen Beschreibungen}
			\item[]  Es ist eine Suche über das Inhaltsverzeichnis bzw. Schlagworte möglich, jedoch nicht über eine Inhaltsbeschreibung
		\end{itemize}
		\item \textbf{Suche nach ähnlichen Publikationen}
		\item[] Eine Suche nach ähnlichen Publikationen ist nicht möglich
		\item \textbf{Suche nach \glqq{}guten\grqq{} Publikationen zu einem Thema}
		\item[] Nein, Publikationen werden nicht bewertet
		\item \textbf{Suche nach \glqq{}passenden\grqq{} Publikationen zu einem Benutzer}
		\item[] Nein
		\item \textbf{Alles im lokalen Bestand oder bibliotheksübergreifend}
		\item[] Entweder innerhalb der Universitätsbibliothek physisch leihbar oder aus dem Netzwerk der Universität erreichbar
		\item \textbf{Erweiterte Metadaten (z.B. eingehende und ausgehende Zitate, Lesefrequenz)}
		\item[] Nein
		\item \textbf{Automatische Zusammenfassung und Aufbereitung von Information aus
		Publikationen}
		\begin{itemize}
			\item \textbf{Beschreibung einer Publikation/einer Reihe mit Schlagwörtern oder als textuelle Zusammenfassung}
			\item[] Es ist nicht erkennbar, ob die Schlagworte automatisch erzeugt werden, aber eine Schlagwortsuche und Schlagworte sind für verschiedene Publikationen vorhanden			\item \textbf{Beantwortung einer konkreten Frage}
			\item[] Nein
			\item \textbf{Zusammenstellung von Argumente für und gegen eine These}
			\item[] Nein
		\end{itemize}
		\item \textbf{Weitere Dienste}
		\begin{itemize}
			\item Logische Verknüpfung von Suchkriterien
			\item Suche für Semesterapparate
		\end{itemize}
	\end{itemize}
	
	\section*{Aufgabe 2}
	\subsection*{a)}
		\begin{itemize}
		\item \textbf{Suche einer bestimmten Publikation}
		\item[] Ja über das Suchfeld, ähnlich wie bei TRICAT
		\item \textbf{Suche nach „passenden“ Publikationen auf Basis von}
		\begin{itemize}
			\item \textbf{Exakten oder inexakten Metadaten}
			\item[] In der erweiterten Suche 
			\item \textbf{Inhaltlichen Beschreibungen}
			\item[] In der erweiterten Suche
		\end{itemize}
		\item \textbf{Suche nach ähnlichen Publikationen}
		\item[] Sobald eine Publikation angeklickt wird, erscheint ein Feld \glqq{}Similar Patents and Articles\grqq{}, welches mittels einer anschaulichen Graph-Visualisierung ähnliche Publikationen anzeigt
		\item \textbf{Suche nach \glqq{}guten\grqq{} Publikationen zu einem Thema}
		\item[] Suchergebnisse lassen sich nach Relevanz sortieren und es sind Bewertungen/Kommentare möglich
		\item \textbf{Suche nach \glqq{}passenden\grqq{} Publikationen zu einem Benutzer}
		\item[] Nein
		\item \textbf{Alles im lokalen Bestand oder bibliotheksübergreifend}
		\item[] Nein
		\item \textbf{Erweiterte Metadaten (z.B. eingehende und ausgehende Zitate, Lesefrequenz)}
		\item[] Es sind beispielsweise bibliometrische Daten, wie zum Beispiel Zitationen oder Anzahl an Downloads vorhanden
		\item \textbf{Automatische Zusammenfassung und Aufbereitung von Information aus
		Publikationen}	
		\begin{itemize}
			\item \textbf{Beschreibung einer Publikation/einer Reihe mit Schlagwörtern oder als textuelle Zusammenfassung}
			\item[] In der erweiterten Suche, gibt es die Möglichkeit eine Full-Text Suche durchzuführen. Auch eine Durchsuchung des Abstracts ist möglich.
			\item \textbf{Beantwortung einer konkreten Frage}
			\item[] Nein
			\item \textbf{Zusammenstellung von Argumente für und gegen eine These}
			\item[] Nein
		\end{itemize}
	\end{itemize}

	Angemeldete Nutzer können Publikationen erwerben und herunterladen. Zusätzlich können Sie diese bewerten und kommentieren. 	
	
	\subsection*{b)}
	Interessant ist vor Allem die Tatsache, dass innerhalb der Publikationen direkt auf Referenzen und Zitationen verwiesen wird. Somit können schnell zum Forschungsthema verwandte Publikationen gefunden werden. Des Weiteren gibt es nach der Suche viele Filtermöglichkeiten. So kann beispielsweise nach einem Autor gesucht werden und es werden im Filter \glqq{}Refine by People\grqq{} Autoren angegeben, welche mit dem gesuchten Autor zusammen publiziert haben.
	
	\subsection*{c)}
	Die bibliometrischen Daten in der oberen rechten Ecke gebe viele Informationen über den Autor an. So kann schnell seine Rolle in der Forschung erkannt werden und es ist über die Anzahl an Zitationen und Downloads eine Tendenz seiner Popularität ersichtlich.
	
	\section*{Aufgabe 3}
	\subsection*{a)}
		\begin{itemize}
			\item \textbf{Suche einer bestimmten Publikation}
			\item[] Die Suche nach bestimmten Publikationen durch Daten wie Titel, Autor, Schlagwörter, u.A. möglich.
			\item \textbf{Suche nach „passenden“ Publikationen auf Basis von}
			\begin{itemize}
				\item \textbf{Exakten oder inexakten Metadaten}
				\item[] Ja
				\item \textbf{Inhaltlichen Beschreibungen}
				\item[] Ja
			\end{itemize}
			\item \textbf{Suche nach ähnlichen Publikationen}
			\item[] Nein.
			\item \textbf{Suche nach \glqq{}guten\grqq{} Publikationen zu einem Thema}
			\item[] Es ist nicht einsehbar ob Publikationen \glqq{} gut\grqq{} oder \glqq{}schlecht\grqq{} sind
			\item \textbf{Suche nach \glqq{}passenden\grqq{} Publikationen zu einem Benutzer}
			\item[] Für angemeldete Nutzer werden Empfehlungen angezeigt, jedoch für nicht-angemeldete Nutzer nicht.
			\item \textbf{Alles im lokalen Bestand oder bibliotheksübergreifend}
			\item[] Einige Publikationen werden nach andere Bibliotheken verlinkt.
			\item \textbf{Erweiterte Metadaten (z.B. eingehende und ausgehende Zitate, Lesefrequenz)}
			\item[] 
			\item \textbf{Automatische Zusammenfassung und Aufbereitung von Information aus
			Publikationen}
			\begin{itemize}
				\item \textbf{Beschreibung einer Publikation/einer Reihe mit Schlagwörtern oder als textuelle Zusammenfassung}
				\item[] Manche Publikationen sind zusammengefasst.
				\item \textbf{Beantwortung einer konkreten Frage}
				\item[] Nein
				\item \textbf{Zusammenstellung von Argumente für und gegen eine These}
				\item[] Nein
			\end{itemize}
			\item \textbf{Weitere Dienste}
		\end{itemize}	

	\subsection*{b)}
		Hier ist es möglich Informationen zu erlangen die genauer die Thematik von bestimmten Ausgaben von Büchern beschreiben und ermöglicht es den Nutzer zu entscheiden ob die Wahl einer dieser sinnvoll wäre. Die Verlinkung zur DBLP erlaubt den Nutzer auch die Möglichkeit andere Publikationen des Autors einzusehen.
		
		Der Hauptunterschied zur ACM ist die Möglichkeit in der ACM die Häufigkeit von Downloads einzusehen.
	\subsection*{c)}
	
	\section*{Aufgabe 4}
	\subsection*{a)}
		Die Lagerung und Pflege der Archiven. Im Vergleich zu digitalen Bibliotheken wird ein enorm größerer Lagerplatz in Anspruch genommen, und hierdurch wird die Pflege der Bücher und anderen Materien deutlich schwieriger gemacht. Es können nur deutlich weniger Informationen auf der gleichen Fläche gelagert werden, und die Pflege erfordert auch eine physische Anwesenheit der Pfleger.
		
		Die Veraltung der Archiven ist auch ein Problem. Bücher und Zeitschriften werden mit ihrem Nutzen immer stärker beschädigt und erfordern nach genug Zeit einen Ersatz. Videos und Filme leiden unter der Veraltung ihrer Technologien. Z.B. sind Videokassetten inzwischen stark veraltet und es wird immer schwieriger Geräte zu finden mit den man diese auch abspielen kann. Dies kann verhindert werden durch Modernisierung auf neuere Formate.
		
	\subsection*{b)}
		Der Verlust von Daten durch Ausfall der Speichermedien ist immer eine Möglichkeit bei digitale Bibliotheken und ist meist nicht so voraussehbar wie bei physischen Medien. Der Ausfall einer Festplatte führt zu Verlust der Daten außer diese Verluste werden durch Datenredundanzen verhindert.
		
		Die Entwicklung von neue Technologien führt auch zu neue Anforderungen an Systeme. Ein Nutzer wird sich heute nicht mehr zufrieden stellen lassen mit der Geschwindigkeit von Suchen die in den 90er Jahren möglich waren und digitale Bibliotheken müssen dementsprechend dauerhaft auf den neusten Stand der Technik gebracht werden.

	\section*{Aufgabe 5}
	\subsection*{a)}
	Eine Digitalisierung von altem Ton- und Videomaterial ist ein sinnvoller Nutzen einer digitalen Bibliothek. Zum einen wird somit der Zugriff für eine deutliche breitere Masse an Menschen ermöglicht und zum anderen werden die Daten so leicht konserviert und somit entfällt die Angst vor Beschädigung der Originale. 
	
	Digitalisierungen von Architektur wären ebenfalls in eine Digital Library denkbar. Zum einen ist es so möglich diese zu betrachten, ohne vor Ort zu sein, zum anderen gibt es auch hier wieder einen Konservierungs-Aspekt.
		
	\subsection*{b)}
	Es gibt keine lokale Einschränkung für den Nutzer. Die Daten sind von überall aus der Welt erreichbar.
	
	Das Durchsuchen der Daten ist deutlich einfacher als in einer klassischen Bibliothek. Dies bietet wiederum eine extreme Zeitersparnis. 
	
	Es können beliebig viele Personen, beliebig Lange, gleichzeitig an demselben Dokument arbeiten. 
	
	\subsection*{c)}
	Bei der Digitalisierung von archäologischen Funden oder Ähnlichem könnte die haptische Beschaffenheit ein Punkt sein, welcher sich derzeit nicht digitalisieren lässt. Durch neue Technologien wie Augmented Reality etc. wurde zwar hier ein wichtiger Schritt gemacht, aber dennoch bietet es noch keine gleichwertige Alternative zum Original. 
	
	Generell könnte es für manche Anwendungsfälle wichtig sein, die Beschaffenheit der Originale zu betrachten, um Rückschlüsse auf eine zeitliche Einordnung zu geben (z .B. in der Religion).
	
	
\end{document}