\documentclass[10pt,a4paper]{article}
\usepackage[utf8]{inputenc}
\usepackage{amsmath}
\usepackage{amsfonts}
\usepackage{amssymb}
\usepackage{graphicx}

\title{Altklausurenfragen Informationsvisualisierung}

\begin{document}
	\maketitle
	\paragraph{Was sind die Phasen des Information Seeking Mantra?}
	\begin{itemize}
		\item Overview - see overall trends
		\item Zoom and Filter - see smaller subset of data, based on criteria
		\item Details on demand - see values of data when interactively selected
	\end{itemize}
	
	\paragraph{Was ist Softwarevisualisierung? + 3 Eigenschaften nennen}
	Visualization of artifacts related to software and its development
	\textbf{Structure, evolution, behavior}
	
	\paragraph{Was sind Isotype?} Isotype if a method of displying (statistical) information in pictogram form. An example from the lecture is the number of ships in the shipping fleets of different countries.
	
	\paragraph{Was sind Sparklines? Wie können sie verwendet werden? + Beispiel zeichnen}
	Sparklines are small, word sized "summaries" of timelines. They can be used to show trends in data. THey could be used to show the trend of global warming or the downward trend of coral reef populations.
	
	\paragraph{Nennen Sie 4 ästhetische Kriterien für das Graphzeichnen}
	Minimize edge crossings, edge length, drawing area, and number of bends; uncover clusters; maximize symmetry and continuity amongst paths
	
	
	\paragraph{Abbildung zu Gestaltgesetz gegeben [war das zur Stetigkeit]. Gesetz nennen und erklären.} Proximity, Similarity, Connectedness, Continuity

	\paragraph{Treemap in Adjazenzmatrix umwandeln}

	\paragraph{Vergleich von geometrischem und semantischem Zoom}
	Geometric zoom is a simple magnification, whereas in semantic zoom, the object being zoomed in to may change the way it is represented in order to show more or less information

	\paragraph{Was ist Edge-Splatting?}
	Edge splatting is using color to encode information on edges with color (especially call frequency in call graphs)

	\paragraph{Vergleich: Vor- und Nachteile von Line Chart (y-Achse: Prozent, x-Achse: Merkmale (statt Zeit)) und Star Plot}
	
	\paragraph{Was ist der Unterschied zwischen Informationsgrafiken und Informationsvisualisierung?}
	Infographics are manually created and serve to communicate a single set of information, while informationvisualization is performed automatically and can be performed on many different data sets
	
	\paragraph{Erläutern Sie wie Tarantula funktioniert. Erfolgreiche und fehlerhafte Testfälle. Gehen Sie auf Helligkeit und Farbgebung ein.}
	Tarantula color encodes the number of times a piece of code is executed when performing tests and records whether lines of code tend to be executed when tests fail or pass. The Redder or more intense the color is the more often a line of code was performed with the result of a failed test 
	
	\paragraph{Was ist eine Evolution Matrix? Welche Phänomene kann man daran beobachten?}
	An evolution matrix shows the evolution of an object-oriented system through the size and number of its classes. Some phenomena that can be identified through the matrix are potentially large leaps in additional functionalities, shown through a large jump in the number of classes. Another could be the redistribution of functionality to other classes through the reduction in the size of one class and the growth of another.
	
	\paragraph{Wie unterscheiden sich unabhängige und abhängige Variablen in einem kontrollierten Experiment?}
	In the most simplistic terms, the independent variable is being changed when performing an experiment and the dependent variable is what is being watched for its behavior as a result of these changes.
	
	\paragraph{Infografik vs. Informationsvisualisierung}
	Infographic - manual, specific, intuitive
	
	\paragraph{Jeweils zwei Vor und Nachteile von 3D Visualisierung anhand von Beispielen erklären}
	
	\paragraph{Was sind die Vorzüge und Nachteile von der Node-Link- und der Adjezenzmatrixdarstellung?}
	
	\paragraph{Erklären Sie Class-Blueprintdiagramme anhand einer Skizze}
	Initialization
	Interface
	Implementation
	Accessor
	Attributes
	
	
	\paragraph{Jackson Diagramm und Konrtollflussdiagram malen}
	
	\paragraph{Was ist ein Forward Memory Referenzmuster? } Subgraph reachable from forward edges of a node. Foreward nodes are referenced objects
	
	\paragraph{Cognitive Dimension Frameworks}
	
	\paragraph{3 Phasen von Visualisierungspipeline} Didn't appear in lecture
	
	
\end{document}