\documentclass[10pt,a4paper]{article}
\usepackage[utf8]{inputenc}
\usepackage{amsmath}
\usepackage{amsfonts}
\usepackage{amssymb}
\usepackage{graphicx}
\usepackage{hyperref}
\author{Aaron Winziers \& Benedikt Lüken-Winkels}
\title{Informations Visualisierung \\ SoSe 19}
\begin{document}
	
	\maketitle
	\tableofcontents
	\newpage
	
	\section{1. Lecture}
	\subsection{Orga}
	\begin{itemize}
		\item Website: st.uni-trier.de/lectures/S19/IV/
		\item Tutorial: TBD (beginning: 22.-26.04.)
		\item Final exam: Do, 11.07. (elfths of July) 12-14 (H12)
	\end{itemize}
	
	\subsection{Visualisation-Basics}
	\paragraph*{Definition: Visualisation} 
	the use of computers or techniques for comprehending data or to extract knowledge from the results of simulations, computations, or measurements (not manually by humans)
	\paragraph*{Definition: Information Visualization}
	the communication of abstract data through the use of interactive visual interfaces.
	\begin{itemize}
		\item Combine different kinds of information in one graphic (geographical, temporal, historical, numeric, etc.)
		\item Sharing and visualising abstract data, without physical representation 
		\item Visualisation is not:
		\begin{itemize}
			\item \textbf{Scientific Visualization} Visualization of data with a concrete physical representation (non-abstract data)
			\item \textbf{Computer Graphics} Technical and mathematical aspects of visualization
			\item \textbf{Graphic Design} Aesthetic graphical representation
		\end{itemize}
		\item \textbf{Example} Treemap
		\begin{itemize}
			\item representation of a hierarchy of a filesystem
			\item no border used for a square (compression)
			\item light effect shows curvature, indicating where the squares/areas end 
			\item $ \Rightarrow $ only 4 pixels needed instead of 9
			\item Several drawbacks (alternative: tree view)
		\end{itemize}
	\end{itemize}
	\paragraph{Abstract Data}
	\begin{itemize}
		\item Text and Tables
		\item Hierarchies and Graphs
		\item Composed data (Multivariate data): Example Napoleon (Slide 1)
		\item Time series: multivariate data with time as a dimension
	\end{itemize}
	\paragraph*{Visualisation process} 
	\begin{itemize}
		\item graphical user interface
		\item interaction to create and manipulate the visualisation (\textbf{Visual steering})
	\end{itemize}
	
	\section{2. Lecture}
	
	\subsection{Diagrams}
	\paragraph{Simple Diagrams}
	\begin{itemize}
		\item Line Charts
		\item Bar Charts
		\item Pie Charts
	\end{itemize}

	\paragraph{Pie charts}
	\begin{itemize}
		\item applicable to part-whole relation
		\item Several issues 
		\begin{itemize}
			\item difficult to compare values within a chart
			\item difficult to compare differences between pie charts
		\end{itemize}
	\end{itemize}

	\paragraph{Other Diagrams}
	\begin{itemize}
		\item \textbf{Timelines} align temporal information along an axis
		\item \textbf{Sparklines} Reduced to show trend and the change of values over time - a sparkline is a small intense, simple, word-sized graphic with typographic resolution.
	\end{itemize}
	
	\subsection{Metaphors and Symbols}
	Make constructs/concepts more accessible/imaginable
	
	\subsection{Symbols/Pictograms}
	highly simplified representation of objects and activities. Very suitable for depicting metaphors
	
	\paragraph{Isotype} using pictograms to convey statistical information. Quantity is better represented by the number of pictograms than by the size of a pictogram.
	
	\subsection{Infographics}
	
	\paragraph{Definition Infographics} Information graphics or infographics are graphic visual representations of information, data or knowledge. These graphics present complex information quickly and clearly, such as in signs, maps, journalism, technicalwriting, and education.
	
	\begin{itemize}
		\item Eyecatcher to get people interested in the presented data
		\item Contain few text
		\item Self-explanatory
		\item Should tell a \textbf{story} $ \Rightarrow $ express an opinion
	\end{itemize}

	\paragraph{Elements of an Infographic}
	\begin{itemize}
		\item Story
		\item Graphics
		\begin{itemize}
			\item Illustrative
			\item Simplified
		\end{itemize}
		\item Text
		\begin{itemize}
			\item Keywords and short texts
		\end{itemize}
		\item Diagrams
		\begin{itemize}
			\item Connected to graphics.
		\end{itemize}
	\end{itemize}

	
	\section{3. Lecture}
	\subsection{Visual Memory}
	\begin{itemize}
		\item The brain fills empty gaps
		\item Distraction by environment (contrast/structure)
		\item $ \Rightarrow $ visual perception is selective
	\end{itemize}
	\subsection{Visual Information Processing}
	3 Phases of processing
	\begin{enumerate}
		\item Simple patterns and colors are recognized
		\item Action system: reflexes
		\item Visual working memory/visual query
	\end{enumerate}
	\subsection*{Human Eye}
	Usage of the properties of visual perception (Anticipation, pattern recognition)
	\begin{itemize}
		\item Eye Tracking (works by measuring the reflection form the eye's curvature)
	\end{itemize}
	
	\subsection{Color Perception}
	3-Color-Theory
	\begin{itemize}
		\item Each color consists of rgb
	\end{itemize}
	Opponent-Color-Theory
	\begin{itemize}
		\item After image effect: color-receptors are getting exhausted, so white cannot be 'produced'
		\item three chemical processes with two opponent colors each 
		\item Color is perceived by the difference between the opponent colors
	\end{itemize}
	$ \Rightarrow $ Color and brightness are relative
	
	\paragraph*{Design Recommendations}
	\begin{itemize}
		\item Emphasize with color
		\item Differences with brightness
		\item Coding of categories: max 6 to 12 different colors
		\item Color scales should vary in color and brighntess 
		\item Color perception depends on culture
		\item Motion to grap attention/indicate a relation
		\item Strong colors/contrast can cause interta (ghost images)
	\end{itemize}
	
	
	\subsection{Preattentive vision}
	\begin{itemize}
		\item Detect patterns before an eye movement
		\item Motion is preattentive
		\item $ \Rightarrow $ Use preattentive patterns to encode information (spot an outlier)
	\end{itemize}
	
	\subsection{Pattern Recognition}
	\begin{itemize}
		\item Edge detection
		\item Simple patterns (detect small distortions)
		\item Complex patterns
		\item Object recognition (compare observation with learned patterns to recognise an object)
	\end{itemize}
	
	
	\subsection{Motion recognition}
	Different elements perform similar motions
	\begin{itemize}
		\item Recognize patterns to identify object
		\item Recognize change after each frame
		\item Movements seem related, when they are in synch
		\item $ \Rightarrow $ Indicate a relation with a synchronous animation 
		\item Motion can induce causality
	\end{itemize}
	
	
	\section{Lecture}
	Visualization of Graphs: \textbf{Graph drawing}
	\paragraph{Application}
	\begin{itemize}
		\item Map-drawing: indicate multiple data sets in one map (London Underground)
		\item Ego(-centric) network: graph with personal connections 
	\end{itemize}
	
	\paragraph{Visual Encoding}
	\begin{itemize}
		\item Thickness, color of edges
		\item Color of nodes
	\end{itemize}
	
	\paragraph{Asthetic Criteria}
	Readability does not induce asthetic
	\begin{itemize}
		\item min edge crossings
		\item min drawing
		\item min edge length
		\item min number of bends
		\item max symmetry
		\item uncover clusters
		\item max continuity amongst paths
	\end{itemize}
	
	\subsection{Layouting algorithms}
	Radial Layout
	\begin{itemize}
		\item fair node weight, every node's representation is equal
		\item lots of edge crossings
		\item applicable, if there is no further info about the data
	\end{itemize}
	Force-Directed Layout
	\begin{itemize}
		\item force edges to a certain length
		\item reorder nodes
		\item try to find equilibrium, where the forces cancel out each other
	\end{itemize}
	Hierarchical Layout
	\begin{itemize}
		\item for cyclic structures: flip the edges that close the cycle while drawing the graph
		\item depth first search provides a topological ordering of the nodes
		\item sort nodes on the lower layer until the bottom is reached, then go back to start
		\item to have a clean layout, put in dummy nodes as a spacer
	\end{itemize}
	Orthogonal Layout
	\begin{itemize}
		\item edges follow grid (orthogonal paths)
		\item shape metrics
		\begin{itemize}
			\item describe the path the edges take by turns
			\item evaluate the paths 
		\end{itemize}
	\end{itemize}
	Edge Bundling
	\begin{itemize}
		\item structured radial layout
		\item bundle edges with the same direction
	\end{itemize}
	
	\subsection{Matrix visualization of Graphs}
	Adjacency Matrix
	\begin{itemize}
		\item indicate an edge in a matrix
		\item uncovering clusters is hard
	\end{itemize}
	\subsubsection*{Layouting}
	Compound graphs
	
	\section{Lecture}
	\subsection{Visualization of dynamic graphs}
	Dynamic graph: sequence of graph states
	\paragraph{Animation} see difference between layout and data changes to preserve the mental map of the graph. Examples:
	\begin{itemize}
		\item TimeLine, horizontal development of the graph, vertical orientation of the graph  
		\item TimeSpiderTrees, cirular layout, each ring is one graph
		\item TimeRadarTrees, cicular layout, outer circles are a representation of the inner. The inner circle shows incoming edges, the outer shows outgoing
	\end{itemize}
	
	\subsection{Multivariate data and time series}
	\paragraph{Boxplots} box showing 50 percent of data, outer borders not standardized
	\paragraph{Fan Chart} wide part shows the mean (similar to the box plot) 
	\paragraph{Histogram} bar represents a range of values (value ragne split into intervals)
	
	\section{Lecture}
	\subsection{Software Visualization: Architecture}
	\paragraph{Pipes and Filters} Input stream providing data, putting it into a pipe of filters
	\paragraph{Layered Systems} Layers provide functionality of upper layers (radial or stacked). Radial: small core, Pyramid: neutral representation 
	\paragraph{Blackboard-driven} Different processes share info on one blackboard
	\subsubsection{Reverse Engineering}
	Create higher level of abstraction for a given system and automatically create architecture visualization. The detection of design patterns is non-trivial. To detect, the program is run and traced. 
	\subsubsection{Enriched Node-Link Diagrams} 
	Visuialize/Encode software metrics. Aggregation of information to simplify.
	\paragraph{Class Blueprint}
	Categorize methods by name and access attributes (public/protected/private...)
	\paragraph{Depenecies Viewer}
	Visualize package graph and dependencies between packages and methods
	\paragraph{Dependency Structure Matrix DSM}
	Detect cycles and indirect cycles with highlighting
	\paragraph{Software Citites and Maps}
	2D plane represents system. Hierarchy shown with trees/dimesions. 3rd dimension can be used to show other metrics, like evolution/age/dependencies
	\paragraph{Summary}
	Ad-hoc diagrams hard to understand without explanation. With reverse engineering automatic creation for specific techniques are possible 
	
\end{document}