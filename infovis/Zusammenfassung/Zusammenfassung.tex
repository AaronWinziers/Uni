\documentclass[10pt,a4paper]{article}
\usepackage[utf8]{inputenc}
\usepackage{amsmath}
\usepackage{amsfonts}
\usepackage{amssymb}
\usepackage{graphicx}
\usepackage{hyperref}
\usepackage{fullpage}
\author{Aaron Winziers \& Benedikt Lüken-Winkels}
\title{Informations Visualisierung \\ SoSe 19}
\begin{document}

\maketitle
\tableofcontents
\newpage

\section{Introduction}
	\subsection{Visualisation-Basics}
	\paragraph*{Definition: Visualisation} 
	the use of computers or techniques for comprehending data or to extract knowledge from the results of simulations, computations, or measurements (not manually by humans)
	\paragraph*{Definition: Information Visualization}
	the communication of abstract data through the use of interactive visual interfaces.
	\begin{itemize}
		\item Combine different kinds of information in one graphic (geographical, temporal, historical, numeric, etc.)
		\item Sharing and visualising abstract data, without physical representation 
		\item Visualisation is not:
		\begin{itemize}
			\item \textbf{Scientific Visualization} Visualization of data with a concrete physical representation (non-abstract data)
			\item \textbf{Computer Graphics} Technical and mathematical aspects of visualization
			\item \textbf{Graphic Design} Aesthetic graphical representation
		\end{itemize}
		\item \textbf{Example} Treemap
		\begin{itemize}
			\item representation of a hierarchy of a filesystem
			\item no border used for a square (compression)
			\item light effect shows curvature, indicating where the squares/areas end 
			\item $ \Rightarrow $ only 4 pixels needed instead of 9
			\item Several drawbacks (alternative: tree view)
		\end{itemize}
	\end{itemize}
	\paragraph{Abstract Data}
	\begin{itemize}
		\item Text and Tables
		\item Hierarchies and Graphs
		\item Composed data (Multivariate data): Example Napoleon (Slide 1)
		\item Time series: multivariate data with time as a dimension
	\end{itemize}
	\paragraph*{Visualisation process} 
	\begin{itemize}
		\item graphical user interface
		\item interaction to create and manipulate the visualisation (\textbf{Visual steering})
	\end{itemize}
	
\section{Infographics}
	
	\subsection{Diagrams}
	\paragraph{Simple Diagrams}
	\begin{itemize}
		\item Line Charts
		\item Bar Charts
		\item Pie Charts
	\end{itemize}
	
	\paragraph{Pie charts}
	\begin{itemize}
		\item applicable to part-whole relation
		\item Several issues 
		\begin{itemize}
			\item difficult to compare values within a chart
			\item difficult to compare differences between pie charts
		\end{itemize}
	\end{itemize}
	
	\paragraph{Other Diagrams}
	\begin{itemize}
		\item \textbf{Timelines} align temporal information along an axis
		\item \textbf{Sparklines} Reduced to show trend and the change of values over time - a sparkline is a small intense, simple, word-sized graphic with typographic resolution.
	\end{itemize}
	
	\subsection{Metaphors and Symbols}
	Make constructs/concepts more accessible/imaginable
	
	\subsection{Symbols/Pictograms}
	highly simplified representation of objects and activities. Very suitable for depicting metaphors
	
	\paragraph{Isotype} using pictograms to convey statistical information. Quantity is better represented by the number of pictograms than by the size of a pictogram.
	
	\subsection{Infographics}
	
	\paragraph{Definition Infographics} Information graphics or infographics are graphic visual representations of information, data or knowledge. These graphics present complex information quickly and clearly, such as in signs, maps, journalism, technicalwriting, and education.
	
	\begin{itemize}
		\item Eyecatcher to get people interested in the presented data
		\item Contain few text
		\item Self-explanatory
		\item Should tell a \textbf{story} $ \Rightarrow $ express an opinion
	\end{itemize}
	
	\paragraph{Elements of an Infographic}
	\begin{itemize}
		\item Story
		\item Graphics
		\begin{itemize}
			\item Illustrative
			\item Simplified
		\end{itemize}
		\item Text
		\begin{itemize}
			\item Keywords and short texts
		\end{itemize}
		\item Diagrams
		\begin{itemize}
			\item Connected to graphics.
		\end{itemize}
	\end{itemize}

	\paragraph{Infographics vs Information Visualization}
	\begin{itemize}
		\item Infographics 
		\begin{itemize}
			\item Manually created
			\item Specially designed for a particular data set
			\item Self-explanatory
		\end{itemize}
		\item Information Visualisation
		\begin{itemize}
			\item Automatically computed
			\item Suitable for a variety of data sets
			\item Not necessarily self explanatory
		\end{itemize}
	\end{itemize}
	

\section{Visual Perception}
	75\% of information is perceived visually
	\subsection{Visual Memory}
		\begin{itemize}
			\item The brain fills empty gaps
			\item Distraction by environment (contrast/structure)
			\item $ \Rightarrow $ visual perception is selective (change blindness)
		\end{itemize}
	\subsection{Visual Information Processing}
		3 Phases of processing
		\begin{enumerate}
			\item Simple patterns and colors are recognized
			\item Action system: reflexes
			\item Visual working memory/visual query
		\end{enumerate}
	\subsection*{Human Eye}
		Usage of the properties of visual perception (Anticipation, pattern recognition)
		\begin{itemize}
			\item Eye Tracking (works by measuring the reflection form the eye's curvature)
		\end{itemize}
	\paragraph{Peripheral Acuity}
		Center of vision:
		\begin{itemize}
			\item In focus
			\item Color and brightness
		\end{itemize}
		\begin{itemize}
			\item Blurry
			\item Only brightness
		\end{itemize}
	
	\subsection{Color Perception}
	3-Color-Theory
	\begin{itemize}
		\item Each color consists of rgb
	\end{itemize}
	Opponent-Color-Theory
	\begin{itemize}
		\item After image effect: color-receptors are getting exhausted, so white cannot be 'produced'
		\item three chemical processes with two opponent colors each 
		\item Color is perceived by the difference between the opponent colors
	\end{itemize}
	$ \Rightarrow $ Color and brightness are relative
	
	\paragraph*{Design Recommendations}
	\begin{itemize}
		\item Emphasize with color
		\item Differences with brightness
		\item Coding of categories: max 6 to 12 different colors
		\item Color scales should vary in color and brighntess 
		\item Color perception depends on culture
		\item Motion to grap attention/indicate a relation
		\item Strong colors/contrast can cause interta (ghost images)
	\end{itemize}
	
	
	\subsection{Preattentive vision}
	\begin{itemize}
		\item Detect patterns before an eye movement
		\item Motion is preattentive
		\item $ \Rightarrow $ Use preattentive patterns to encode information (spot an outlier)
	\end{itemize}
	
	\subsection{Pattern Recognition}
	\begin{itemize}
		\item Edge detection - Differences in brightness, color, texture or motion
		\item Simple patterns (detect small distortions)
		\item Complex patterns
		\item Object recognition (compare observation with learned patterns to recognise an object)
	\end{itemize}
	
	\subsection{Motion recognition}
	Different elements perform similar motions
	\begin{itemize}
		\item Recognize patterns to identify object
		\item Recognize change after each frame
		\item Movements seem related, when they are in synch
		\item $ \Rightarrow $ Indicate a relation with a synchronous animation 
		\item Motion can induce causality
	\end{itemize}

	\subsection{Gestalt Psychology}
	\begin{itemize}
		\item \textbf{Proximity} - Elements which are placed close to each other are perceived as a group.
		\item \textbf{Similarity} - Similar elements (form, color) are perceived as a group.
		\item \textbf{Connectedness} - Connected elements are perceived as one object
		\item \textbf{Continuity} - For humans it is easier to group continuous elements than elements with abrupt changes of direction.
	\end{itemize}

	\subsection{Three-Dimensional Perception}
		\paragraph{Reconstruction of depth information}
		\begin{itemize}
			\item Stereoscopic vision (in particular at close range)
			\item Occlusion of objects
			\item More depth cues: depth of field, perspective, shadow, scale, contrast, motion parallax (how near and far objects will move across the retina of an eye as we move along in the world)
			\item Prior knowledge
		\end{itemize}
	
\section{Visualizations of Hierarchies}
	Hierarchy = Tree
	\subsection{Node-Link}
		\paragraph{Types}
		\begin{itemize}
			\item Phylogenetic Tree
			\item Radial Tree
			\item Cone Trees
		\end{itemize}
		\paragraph{Advantages}
		\begin{itemize}
			\item Intuitive
			\item Hierarchy immediately recognizable
			\item Very flexible layout
		\end{itemize}
		\paragraph{Disadvantages}
		\begin{itemize}
			\item Edges require space
			\item Difficult to add labels
			\item Degenerated trees are difficult to represent
		\end{itemize}
	
	\subsection{Indented Outline Plots}
		\paragraph{Examples/Types}
		\begin{itemize}
			\item Windows explorer
			\item XML File 
		\end{itemize}
		\paragraph{Advantages}
		\begin{itemize}
			\item Very readable
			\item Easy to add labels
			\item Familiar; used daily by many people (file explorer)
			\item Degenerated trees can be represented
			\item Hierarchy is well recognizable
		\end{itemize}
		\paragraph{Disadvantages}
		\begin{itemize}
			\item Inner nodes require space
			\item Somewhat inflexible layout
		\end{itemize}
	
	\subsection{Icicle Plots}
	\paragraph{Examples/Types}
	\begin{itemize}
		\item InfoVis Toolkit
		\item Sunburst
		\item Hierarchical Edge Bundles
	\end{itemize}
	\paragraph{Advantages}
	\begin{itemize}
		\item easy to add labels
		\item hierarchy is well	recognisable
		\item flexible layout
		\item uses screen space efficiently
	\end{itemize}
	\paragraph{Disadvantages}
	\begin{itemize}
		\item somewhat less intuitive
		\item available width for children restricted by the width of of their parents.
	\end{itemize}

	\subsection{Treemap}
	\paragraph{Examples/Types}
	\begin{itemize}
		\item Treemap
		\item Information Pyramids
		\item CodeCity
	\end{itemize}
	\paragraph{Advantages}
	\begin{itemize}
		\item area of leaf nodes can be used
		\item can fill arbitrary shapes e.g. Voronoi treemaps)
		\item inner nodes require less space
		\item edges require (almost) no space
	\end{itemize}
	\paragraph{Disadvantages}
	\begin{itemize}
		\item less intuitive
		\item hierarchical structure difficult to recognise
		\item difficult to add labels
	\end{itemize}

	\subsection{Empirical Study of Efficacy}
		\paragraph{Recommended}
		\begin{itemize}
			\item Node-Link Diagrams
			\item Icicle Plots
			\item (Indented Outline)
		\end{itemize}
		\paragraph{Questionable}
		\begin{itemize}
			\item Treemap
			\item radial layouts
		\end{itemize}
		\paragraph{Conclusion}
		\begin{itemize}
			\item Empirical evaluation is just beginning
			\item More research is needed to make well-founded design recommendations
			\item There is also a lack of domain-specific results.
		\end{itemize}
	
\section{Visualization of Graphs}
	\paragraph{Graph Drawing} - The art of drawing a diagram of a graph to facilitate understanding of relations between objects
	\paragraph{Application}
	\begin{itemize}
		\item Map-drawing: indicate multiple data sets in one map (London Underground)
		\item Ego(-centric) network: graph with personal connections 
	\end{itemize}
	
	\paragraph{Visual Encoding}
	\begin{itemize}
		\item Thickness, color of edges
		\item Color of nodes
	\end{itemize}
	part
	\paragraph{Aesthetic Criteria}
	Readability does not induce aesthetic
	\begin{itemize}
		\item Minimize edge crossings
		\item Minimize drawing area
		\item Minimize edge length
		\item Minimize number of bends
		\item Maximize symmetry
		\item Uncover clusters
		\item Maximize continuity amongst paths
	\end{itemize}
	
	\subsection{Layouting algorithms}
		Radial Layout
		\begin{itemize}
			\item fair node weight, every node's representation is equal
			\item lots of edge crossings
			\item applicable, if there is no further info about the data
		\end{itemize}
		Force-Directed Layout
		\begin{itemize}
			\item force edges to a certain length
			\item reorder nodes
			\item try to find equilibrium, where the forces cancel out each other
		\end{itemize}
		Hierarchical Layout
		\begin{itemize}
			\item for cyclic structures: flip the edges that close the cycle while drawing the graph
			\item depth first search provides a topological ordering of the nodes
			\item sort nodes on the lower layer until the bottom is reached, then go back to start
			\item to have a clean layout, put in dummy nodes as a spacer
		\end{itemize}
		Orthogonal Layout
		\begin{itemize}
			\item edges follow grid (orthogonal paths)
			\item shape metrics
			\begin{itemize}
				\item describe the path the edges take by turns
				\item evaluate the paths 
			\end{itemize}
		\end{itemize}
		Edge Bundling
		\begin{itemize}
			\item structured radial layout
			\item bundle edges with the same direction
		\end{itemize}

	\subsection{Matrix visualization of Graphs}
		Adjacency Matrix
		\begin{itemize}
			\item indicate an edge in a matrix
			\item uncovering clusters is hard
		\end{itemize}
	
	\subsubsection*{Layouting}
		Compound graphs

	\subsection{Visualization of dynamic graphs}
		Dynamic graph: sequence of graph states
		
		\subsection{Approaches to dynamic graphs}
		
		\paragraph{Animation} Animation of the sequence of graphs
		\begin{itemize}
			\item \textbf{Local goal} - Optimal graph layout
			\item \textbf{Preserving the mental map}
		\end{itemize}
		
		\paragraph{Time Line} - Visualization of the sequence of graphs as a series of
		static images along a time line. Examples:
		
		\begin{itemize} 
			\item TimeSpiderTrees, cirular layout, each ring is one graph
			\item TimeRadarTrees, cicular layout, outer circles are a representation of the inner. The inner circle shows incoming edges, the outer shows outgoing
		\end{itemize}
		
\section{Multivariate data and time series}
	\paragraph{Multivariate Data}
		\begin{itemize}
			\item Several variables/dimensions per
			object/observation
			\item Types of variables - numeric, categorial
			\item Easy to represent in a table
		\end{itemize}
	
	\paragraph{Descriptive Statistics}
	\begin{itemize}
		\item Mean
		\item Median
		\item Quartile
		\item Mode
		\item Standard Deviation
		\item Standard Error
	\end{itemize}

	\subsection{Graph types}
		
		\paragraph{Boxplots} box showing 50 percent of data, outer borders not standardized
		\paragraph{Fan Chart} wide part shows the mean (similar to the box plot) 
		\paragraph{Histogram} Frequency distribution shown as bar chart (value range split into intervals)
		
		\paragraph{Extended table} - With color coding, bars and icons
		\paragraph{Sparklines in tables}
		
		\paragraph{Scatterplot}
		\paragraph{Scatterplot matrix} - creating multiple 2-dimensional scatterplots in a matrix
		
		\paragraph{Parallel Coordinates}
		\paragraph{Star Plots} - radial variant of parallel coordinates

\section{Software Visualization: Code}

	\paragraph{Software visualization} - Visualization of artifacts related to software and its development process
	\begin{itemize}
		\item Structure
		\begin{itemize}
			\item Software architecture
			\item Dependencies between software artifacts
			\item Data structures
		\end{itemize}
		\item Behavior
		\begin{itemize}
			\item Execution of an algorithm
			\item Runtime behavior
			\item Program state 
		\end{itemize}
		\item Evolution
		\begin{itemize}
			\item Development history of a software system
			\item (Sequences of) source code changes
			\item Team buildung and development
		\end{itemize}
	\end{itemize}

	\paragraph{Pretty Printing}
	\begin{itemize}
		\item Line breaks to discern statements
		\item Indentation to make the structure more explicit
	\end{itemize}

	\paragraph{Syntax Highlighting}

\section{Interaction}
	\paragraph{Shneiderman’s Taxonomy of Information Visualization Tasks}
	\begin{itemize}
		\item \textbf{Overview}: see overall patterns, trends
		\item \textbf{Zoom}: see a smaller subset of the data
		\item \textbf{Filter}: see a subset based on values, etc.
		\item \textbf{Details on demand}: see values of objects when interactively selected
		\item \textbf{Relate}: see relationships, compare values
		\item \textbf{History}: keep track of actions and insights
		\item \textbf{Extract}: mark and capture data
	\end{itemize}

	\paragraph{Shneiderman‘s Information-Seeking Mantra} - Overview first, zoom \& filter, then details-on-demand

	\paragraph{Categories of Interaction Techniques}
	\begin{itemize}
		\item \textbf{Select} - Mark something as interesting
		\item \textbf{Explore} - Show me something else
		\item \textbf{Encode} - Show me a different visual representation
		\item \textbf{Reconfigure} - Change the spatial arrangement
		\item \textbf{Abstract/Elaborate} - Show me more or less detail
		\item \textbf{Filter} - Show me something conditionally
		\item \textbf{Connect} - Show me related items
	\end{itemize}

	\paragraph{Standard vs. Semantic Zoom}
	\begin{itemize}
		\item \textbf{Geometric Zooming (Standard)} - View depends on the physical properties of the presented object
		\item \textbf{Semantic Zooming} - A different visual representation is chosen depending on what meaning of the presented object should be preserved.
	\end{itemize}


\section{Software Visualization: Architecture}
	\paragraph{Software Architecture} - Architecture is the fundamental organization of a system embodied in its components, their relationships to each other and to the environment and the principles guiding its design and evolution
	
	\subsection{Common Architectures}
		
		\paragraph{Pipes and Filters} Input stream providing data, putting it into a pipe of filters
		
		\paragraph{Layered Systems} Layers provide functionality of upper layers (radial or stacked). Radial: small core, Pyramid: neutral representation 
		
		\paragraph{Blackboard-driven} Different processes share info on one blackboard
		
		\paragraph{UML}
		
	\subsection{Reverse Engineering}
	Reverse engineering is the process of analyzing a subject system to create representations of the system at a higher level of abstraction. $\rightarrow$ used for automatically creating architecture visualizations
	
	\subsection{Enriched Node-Link Diagrams} 
	Visuialize/Encode software metrics. Aggregation of information to simplify.
	
	\paragraph{Software Metrics} A software metric is a measure of some property of a
	piece of software or its specifications.
	\begin{itemize}
		\item software metrics provide additional information
		\begin{itemize}
			\item automatic computation
			\item usually: multivariate data
		\end{itemize}
		\item may increase understanding, help to find problems
	\end{itemize}
	
	\paragraph{Class Blueprint}
	Categorize methods by name and access attributes into:
	\begin{itemize}
		\item \textbf{Initialization} - methods with substring "init" or "initialize", constructors
		\item \textbf{Interface} - public or protected methods, only invoked by init layer within the same class 
		\item \textbf{Implementation} - private methods invoked by other methods in the same class
		\item \textbf{Accessor} - methods to get and set the values of attributes (getter/setter)
		\item \textbf{Attributes} - all attributes of the class
	\end{itemize}
	
	\paragraph{Depenecies Viewer}
	Visualize package graph and dependencies between packages and methods
	
	\paragraph{Dependency Structure Matrix DSM}
	Detect cycles and indirect cycles with highlighting

	\paragraph{Software Cities and Maps}
	2D plane represents system. Hierarchy shown with trees/dimesions. 3rd dimension can be used to show other metrics, like evolution/age/dependencies
	
	\paragraph{Summary}
	Ad-hoc diagrams hard to understand without explanation. With reverse engineering automatic creation for specific techniques are possible 
	
\section{Lecture}
	\subsection{Dynamic Program Visualization}
		
		\paragraph{Dynamic Data Acquisition} invasive method, monitoring the behavior of a program before/after each instruction. Might alter the program execution.
		\begin{itemize}
			\item Instrumentation
			\begin{itemize}
				\item before/after each instruction
				\item at certain program points
				\begin{itemize}
					\item before/after loops
					\item before/after method calls
					\item defined by user ($\rightarrow$ interesting events)
				\end{itemize}
				\item data structures
				\begin{itemize}
					\item Whenever data is changed (daemon, observer)
				\end{itemize}
			\end{itemize}
			\item Parallel thread, which reads memory
			\item Capture messages (for distributed programs)
			\item Virtual Machine/Interpreter
			\item Special Purpose Hardware ($\rightarrow$ embedded systems)
		\end{itemize}
	
		\paragraph{What data is to be captured?}
		\begin{itemize}
			\item Program position (PC, called method, line number in source
			code)
			\begin{itemize}
				\item Problem for compiled programs: mapping machine instructions to line numbers in the source code
			\end{itemize}
			\item Values of program variables
			\item Heap contents of the program
			\item For messages:
			\begin{itemize}
				\item Point in time (enables temporal ordering of messages, which have been captured at different computers, Problem: local vs. global time)
			\end{itemize}
		\end{itemize}
	
		\paragraph{Architectures for Algorithm Animation}
		\begin{itemize}
			\item Ad Hoc Visualization and Libraries
			\begin{itemize}
				\item  Don‘t use a tool at all. Implement everything from scratch.
				\item Use libraries with graphical abstractions, control-elements, etc.
			\end{itemize}
			\item Special data types
			\begin{itemize}
				\item Program the algorithm with datatypes which have built-in visualizations
			\end{itemize}
			\item Post-Mortem Visualisierung
			\begin{itemize}
				\item Record an event log or animation script during the execution of the algorithm.
				\item Animation after the execution of the algorithm.
			\end{itemize}
			\item Interesting Events
			\begin{itemize}
				\item Annotate interesting program points
				\item Send events to concurrently executed animation (view)
			\end{itemize}
			\item Declarative
			\begin{itemize}
				\item Separation of annotation and algorithm
				\item Demon monitors state changes and visualizes the state
			\end{itemize}
			\item Semantics-Directed
			\begin{itemize}
				\item automatic visualization by visual interpreter or debugger for the programming language
			\end{itemize}
		\end{itemize}
		
		
	\subsection{Visual Debugging}
		Slices are parts/slices of the huge dependency graph in a program 
		\paragraph{Static Slice} How is a variable changed by other code points. Slice is a small part
		\paragraph{Dynamic Slice} 
		\paragraph{Execution Slices} Sequence of program points.
		
		\paragraph{Dice} Difference of two Slices.
		
		\paragraph{X-Slice} (Heuristic) Compare a run with failing and compiling input. Only the failing program points are highlighted. Color coding coverage data by failure propability and evidence for failure.
		
		\paragraph{Test Blueprint} Highlight non-executed program points in the Class Blueprint. 
		
	
	\subsection{Software Evolution}
		aka Software Development Process $ \Rightarrow $ Software changes in its lifetime.
		
		\paragraph{Software Archive} version control/collection of the history of a program of any kind. 
		
		\paragraph{Color-coding}
		\begin{itemize}
			\item Line Representation: indentation/different metrices
			\item Code Age: when was a file/line changed
			\item Pixel Representation
			\item Version-specific Code: highlight eg platform specific code
			\item Depth of nested blocks
			\item CVS Scan: different versions for a file with LOC as bar height.
		\end{itemize}
		
		\paragraph{Evolution Matrix} Classes are represented as boxes. Box height and width encode a certain metric. $ \Rightarrow $ No insight on program structure
		
		\paragraph{Call Graph} Which function calls wich function (low level info). Encode program structure. \textbf{Edge splatting} (the more often an edge is drawn the more intense it color gets) shows call clusters.
		
	\subsubsection{Visual Data Mining in Software Architecture}
		\paragraph{Data Mining Process}
		Starting with a version control program (git)
		\begin{enumerate}
			\item Analysis
			\item Extraction
			\item Data Mining
			\item Visual Data Mining
		\end{enumerate} 
		
		
		\paragraph{Coupling}
		\begin{itemize}
			\item Evolutionary Coupling artifact are related, when they are changed together.
			\item Logical Coupling artifacts are related, when they are programmatically calling each other.
		\end{itemize}
			
\end{document}