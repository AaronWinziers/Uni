\documentclass[10pt,a4paper]{article}
\usepackage[utf8]{inputenc}
\usepackage{amsmath}
\usepackage{amsfonts}
\usepackage{amssymb}
\usepackage{graphicx}
\author{Aaron Winziers}
\title{InfoVis Summary}
\begin{document}
	\section{Introduction}
		\paragraph{Information visualization} is the communication of abstract data through the use of interactive visual interfaces.
		
		\paragraph{Abstract Data} examples:
			\begin{itemize}
				\item Text and Tables
				\item Hierarchies and Graphs
				\item Multivariate Data
				\item Time Series
			\end{itemize}
		
		\paragraph{Visualization} is the use of computers or techniques for comprehending data or to extract knowledge from the results of simulations, computations, or measurements
		
		\paragraph{Demarcation of research areas:}
			\begin{itemize}
				\item \textbf{Scientific Visualization} Visualization of data with a concrete physical representation (non-abstract data)
				\item \textbf{Computer Graphics} Technical and mathematical aspects of visualization
				\item \textbf{Graphic Design} Aesthetic graphical representation
			\end{itemize}
		
		
	\section{Infographics}
		\subsection{Diagrams}
			\paragraph{Simple Diagrams}
				\begin{itemize}
					\item Line Charts
					\item Bar Charts
					\item Pie Charts
				\end{itemize}
			
			\paragraph{Pie charts are trash}
				\begin{itemize}
					\item Pro - Works well for comparing the size of a single category with the total size. 
					\item Contra - Difficult to compare size of different categories
					\item Alternatives - Bar chart, stacked bar chart
				\end{itemize}
			
			\paragraph{Timelines} align temporal information along an axis
		
			\paragraph{Sparklines} a sparkline is a small intense, simple, word-sized graphic with typographic resolution.
		
		\subsection{Metaphors and Symbols}
			\paragraph{Metaphor} a figure of speech that, for rhetorical effect, directly refers to one thing by mentioning another. It may provide clarity or identify hidden similarities between two ideas
			
			\paragraph{Metaphors in Visualization}
				\begin{itemize}
					\item Cities
					\item Trees
					\item Animals
				\end{itemize}
			
			\paragraph{Symbols/Pictograms}
				\begin{itemize}
					\item Highly simplified pictorial representation of objects or activities.
					\item Very suitable for depicting metaphors
				\end{itemize}
			
			\paragraph{Isotype} (Ethnicities by country - funny racist chart) using pictograms to convey statistical information. Quantity is better represented by the number of pictograms than by the size of a pictogram.
			
		\subsection{Infographics}
			\paragraph{Elements of an Infographic}
				\begin{itemize}
					\item Story
					\item Graphics
					\subitem Illustrative
					\subitem Simplified
					\item Text
					\subitem Keywords and short texts
					\item Diagramme
					\subitem Connected to graphics
				\end{itemize}
			
			\paragraph{Definition Infographics} Information graphics or infographics are graphic visual representations of information, data or knowledge. These graphics present complex information quickly and clearly, such as in signs, maps, journalism, technicalwriting, and education.
		
			\paragraph{Infographics vs Information Visualization}
				
		
		
	\section{Visual Perception}
	\section{Visualization of Hierarchies}
	\section{Visualization of Graphs}
	\section{Multivariate Data and Time Series}
	\section{Software Visualization}
	\section{Interaction}
	\section{Software Visualization Part 2: Architecture}	
	\section{Software Visualization Part 3: Behavior}
\end{document}