\documentclass[10pt,a4paper,parskip=quarter ]{article}
\usepackage[utf8]{inputenc}
\usepackage[ngerman]{babel}
\usepackage{amsmath}
\usepackage{amsfonts}
\usepackage{amssymb}
\usepackage{graphicx}
% \usepackage{fullpage}
\usepackage{xcolor}
\usepackage{float}
\usepackage{graphicx}
\usepackage{hyperref}

\usepackage{fancyhdr}

\pagestyle{fancy}
\rhead{Aaron Winziers - 1176638, Benedikt Lüken-Winkels - 1138844}

\author{Aaron Winziers - 1176638 \\
Benedikt Lüken-Winkels - 1138844}
\title{Information Visualization SS 2018\\\LARGE{Assignment 8}}

\begin{document}
	\maketitle
	\section*{Excersize 8.1}
	
	\section*{Exercise 8.2}
	The city metaphor could additionally be used to model any number of different complex and geographically widespread organizations.
	
	A large, complex company with multiple departments could have each department represented by a different building. Depending on the granularity of the metaphor and the complexity of the company, each city block could represent a department, with the buildings "built" on the block representing the individual teams within the department. The size of the buildings would not be restricted to representing just the size of the teams; they could also represent the revenue brought in by the teams or departments (though this could become more complicated with teams performing at a loss), or even their operating costs.
	
	A widespread organization, such as a motorcycle club could also easily be represented. Here, the buildings would represent the different chapters of the club, again with the height of the buildings potentially representing a number of things. In this example, the number of members could determine the height, however, a more fun variation could be the average loudness of the members' motorcycles, the average number of miles ridden by the members of the club, or the number of fights the clubs involved themselves in. The mapping of the buildings could be used to show the rough geographic distances between the different chapters.
	
\end{document}